
\chapter{写给新生:从入门到放弃}

\section{大一新鮮人的履歷表}
% http://homepage.ntu.edu.tw/~ylwang2008/a-undergraduate.htm


凌晨兩點,還在看朋友的blog。
「你明天不是要去旅行嗎?怎麼都沒看你在打包行李?」 媽媽有點著急。
「啊~ 別擔心啦!一下子就好了啦!」頭也不回一下。
又過了一陣子,回覆完最後一個訊息, 從MSN登出。終於開始翻箱倒櫃,把東西塞進包包。

「輕便雨衣?家裡沒有耶。沒關係,雖然貴了點,但便利商店就有。」
「對了,夜遊要用手電筒!算了,跟朋友共用就好。」
「隱形眼鏡藥水不夠了,不知道一大早哪裡有賣?」
「啊!數位相機的記憶卡壞了,一直說要買新的,都還沒買。」
「駕照,駕照,怎麼找不到機車駕照?咦? 床底下怎麼有漫畫?
忘記拿去還了!完了完了,又要被罰錢了!
該死!我的駕照還押在漫畫出租店!慘了慘了!」

*****************************************************************************************

這個場景,應該不會很陌生吧?
我相信你一定聽過,甚至親身經歷過類似的故事。
行前不早早準備,小則花錢解套,大則耽誤整個行程。

你知道嗎?有許多大學生,對人生的態度也是這樣。
糊里糊塗地,等到即將畢業踏上旅程的那一刻,才驚覺自己還沒打包好行李。

無助、徬徨、悔不當初......各種情緒湧上心頭,卻無法改變什麼。
更討厭的是,自己已經夠心慌了,
旁邊的人還要落井下石地碎碎唸:「你看看,我就跟你說嘛,早知道......」

然而,不可否認的是, 如果在剛知道要出門旅行的時候,就先準備好一份清單,
這個故事的結局,會很不一樣。
缺的項目,可以提早準備,壞的物品可以提早修補。

以大學生來說,這個清單就叫做「履歷表」。
不管你在大學畢業後要找工作,還是考研究所, 總免不了要交出一張履歷表,告訴別人你的包包裡裝了什麼。 既然早寫晚寫,都要準備這份履歷表,何不及早從大一就開始呢?

舉例來說,你希望在畢業之後,進入外商銀行工作。 那麼這份清單上所列的項目,可能包括了-- 優秀的成績、社團管理經驗、金融證照、英文能力等等。 那麼,你就有四年的時間,來準備這些項目。

或許你的英文很破,但你可以長期抗戰。 你可以自由規劃,例如:大一下考全民英檢,大二時參加中英文雙語教會, 升大三暑假去遊學,大四考拖福......

你可以對照自己目前的履歷表,以及理想狀態的履歷表, 量身訂作出短期、中期,與長期的補強計畫,並隨時調整。 就算英文底子再差,只要提早準備,效果絕對遠勝於畢業時,才臨時抱佛腳,勝過讀什麼『30天英語速成法』之類的書。

所以,我鼓勵各位大一新鮮人,在讀到這篇文章後的一週內,草擬一份大學畢業時的履歷表。 看看自己目前缺了哪些項目,安排一些計畫來加強。 然後,找兩三位好朋友一起分享討論,並約定大學這四年間,要互相鼓勵與監督。

如果,你想不到履歷表上該列哪些項目,請找該領域的學長姊聊聊,並上求職網站逛逛。
這份屬於你的履歷表,沒有正確解答,也沒有公式可套。 但只要有心規劃,一定可以寫得出來!

願各位都能--「積極認真地打包行李, 滿懷信心地出門旅行!」


\section{如何在学术领域一事无成}



\section{研究生「論文寫作症候群」}

自我檢查法 
檢視下列題目,自我檢查是否罹患「論文寫作症候群」: 
(每題一分,對問題感到特別心有戚戚焉者,可以加重計分:「同感」加五分;「相當同感」加十分;「天哪!你真是我的知音」,可自行加分無上限。)

01.一星期中有三天以上,下午兩點以後才吃第一頓飯。
02.對你而言,失眠乃是家常便飯的事。
03.寫作論文時,鍵盤與手指頭曾出現過「同極互斥」的現象。但打B或玩game時此現 象便消失……
04.曾有過異常昏睡的紀錄。
05.對論文以外的一切事物都感到興趣。
06.曾為論文的事去廟裡拜拜達兩次以上。
07.和同時寫作論文的同學達到前所未有的心靈相通狀態。
08.寫論文期間某種特殊技能大進。(如玩game、聽某種類型音樂、玩樂器、塗指甲油 等等……)
09.早已脫離青春期的你,又開始冒起青春痘。 
10.曾在網路上搜尋或詢問過:「請推薦不錯的眼科,謝謝!」這類的問題。 
11.曾不止一次想像過口試時的情形。
12.每天晚上都會告訴自己:「我會撐過去的。」
13.謝絕訪客的原因是房間實在凌亂得讓自己都無法面對。
14.對英文F開頭以及S開頭的字感到空前親切。
15.某些癮頭使你認知到自己身體的無限可能或極限。(煙癮、酒癮、咖啡癮……)
16.對自己的能力以及適不適合走學術這條路的問題,有過前所未有的深刻體會及考慮。 

成績計算: 
得分1~6分的人: 
少來了!你不是症狀嚴重到自己都不自知,就是計分方式沒看清楚!要不然就是你是個天生吃這行飯的人,未來學術界的閃亮明星!如果是後者的話,請您留下您的聯絡方式,未來還需要您給我好好「照顧」一下哪! 
得分6~30分的人: 
其實,你的症狀算還是輕微的!不過就是午夜夢迴的時候會想:為什麼我沒那麼嚴重呢?看到許多同學寫得那麼辛苦,我什麼沒有?為什麼沒有?難道…….我真的不是這塊料?聽到同學們談論他們的症狀,你隱隱覺得,罹患「論文寫作症候群」好像是一種身段,一件值得炫耀的事。所以,你暗暗下定決心,也要點根煙,小小聲地罵幾句shit,然後……然後,就回頭繼續敲打你的論文去啦! 
得分30以上的人: 
麥哥共啊!我瞭解你的心情……誰叫我們都想寫出什麼驚天地泣鬼神、轟動學術界的論文!但事實是,寫好擺在國家圖書館或社資中心的架上之後,若真有機會被人拿起翻閱,充其量,也只是,驚動一些灰塵罷了!當然,除了看開一點外,你還有一種自我解救方法:即所謂的「團體治療」法,也就是把你的種種症狀寫出來,投稿到「論文寫作症候群」網站,大家互相吐吐苦水,也是有某種同舟共濟的效果啦!

註:本文為網路流傳之文件,来源不可考。




\section{ 硕博研究生病情诊断}
\label{phd_diagnostic}
文章转自台湾江源慎网络作品"一起來认识「論文寫作症候群」"。\url{https://mypaper.pchome.com.tw/ehara/post/3741124}


研究生發病期...請參照說明,以鑑定您的另一半身何期..

潛伏期:不定。
傳染性:無。
感染人數:全台每年約十至十二萬人。
並無性別年紀的區分,只要身份為研究生,通常都會罹病,極少倖免。
關於本症候群,約莫能粗分為三大時期。第一期:
通常發生於準備開始寫論文時,有以下幾種指標型症狀。
恭喜!你已經進入本症候群的初期狀況!

症狀:
(1) 覺得自己挑的論文題目簡直棒呆了,這麼棒的題目,怎麼沒人發現!?天助我也!

(2) 覺得自己天縱英才,就算沒能寫出絕世論文,也將撇出矚目佳作。

(3) 作研究回顧時,覺得他人的研究成果不過爾爾,狠狠地痛批別人一頓,快意快意。

(4) 能夠草擬出自認為天衣無縫的研究計畫以及進度表,並能與他人侃侃而談,意氣風發。


第二期:
這一期通常發生於論文開始動筆之後,發病時間很長,還能細分為三個階段,各有不同,茲
說明如下:

第一階段:
通常發生在寫大綱後動筆前。
症狀:
(1) 開始瞭解,連寫大綱都不是件容易的事情。

(2) 發現資料居然出奇的難找,開始懷疑有寫出絕世論文的可能,而且懂得為何別人都沒
挑這個題目的原因。

(3) 發現懂的東西實在太少,終於瞭解「書到用時方恨少」的道理,但是已經徹底來不及
了。

(4) 根本不敢動筆,「準備不週」的恐懼隨時縈繞心頭,怎麼寫都覺得不對勁,怎麼看都
覺得不爽快。
(5) 容易累,容易疲憊,容易想睡。

(6) 開始害怕老闆的電話,覺得實驗室是個麻煩的地方。

第二階段:
通常發生在實際動筆之後。
症狀:
(1) 對任何跟論文相關的東西絲毫沒有興趣,但是對於論文以外的東西非常感興趣,通常
會附帶學會第二專長,像是:室內布置、煮咖啡、打毛線、美容、占星、打PSII、唱KTV、
學怎麼使用Lovema。

(2) 開始思考一些從來不會思考的問題,像是:「我活著是要幹嘛的?」、「這篇論文的
價值在哪裡?」、「我這樣寫會不會被告?」
(3) 開始制訂玩耍計畫,小至下午茶,大至海外旅遊,通通有興趣。

(4) 對於「計畫跟不上變化」這句話,不僅認同,簡直是實地體驗。

(5) 弄不清晝夜晨昏,醒來通常已經是午後,甚至天黑,作息混亂到根本不知道該吃
早餐還是午餐。

(6) 學會快速轉台,甚至能清楚的講出每天電視的節目單。

(7) 體態有嚴重的改變,可能狂肥,可能狂瘦。

(8) 逃避所有Meeting,能躲老闆躲多久,就躲多久。

(9) 站在鏡子前,會驚訝的發現:有個邋遢的野人在鏡子裡頭瞪著妳看。

(10) 非常希望一覺醒來,發現自己還沒念研究所,如果能的話,希望把所有東西都丟了
。通常發生在論文繳交期限前一個月。

症狀:
(1) 完全沒退路的狀況下,只好硬著頭皮上,對於能寫出啥子東西,完全沒信心,瞭解「
不忍卒睹」的真意。

(2) 面對胡言亂語的內容,開始努力說服自己抄襲。

(3) 挑戰身體極限的活動逐步出現,像是:「連續30小時不睡覺!」、「挑戰一天打一萬
五千字」。

(4) 發現跟老闆Meeting是天下最恐怖的事情,老闆不用作啥,看著初稿不說話就夠嚇人
了
(5) 開始使用Word裡,「邊界」、「行距」跟「字型大小」的功能,努力擴大篇幅。

(6) 抗壓性極低,任何風吹草動都令人抓狂。

(7) 覺得「能寫完論文」就是一件很了不起的事情了,別想說要寫得多好,寫完就好。

(8) 根本不知道外頭天氣為何,伴著你的就是鍵盤跟螢幕。

(9) 覺得看這本論文的人,全世界加一加大概不會超過五個。

(10) 本來想感謝一堆人的,但是已經沒氣力寫謝辭了。

第三期:
這一期會發生在口試本交出後,等待口試前。

症狀:
(1) 對於交出的東西害怕不已,用便利貼貼住自己覺得有問題的地方,竟然幾乎貼滿整本
。

(2) 根本不知道口試會發生啥子事情,驚愕不已,鎮日睡不好。

(3) 忽然間意識到房間竟然亂到不行,非常懷疑自己怎麼能在這樣的地方活著。
(4) (到庙里)拜拜的時間變多,希望口試時遇到好人。

(5) 開始思考未來該怎麼辦?懷疑自己根本不適合學術圈,並嘗試詢問雞排店以及飲料吧
的加盟辦法。

(6) 開始煩惱該花多少錢印論文。

(7) 對於自己曾經堅持把一部無聊的連續劇從頭看到完,驚訝不已。
(8) 忽然間想起另一半的存在,不過,另一半可能已經不見了。

(9) 覺得很對不起「樹木」。

病癒:

口試通過,拿到畢業證書後的後遺症,延續期間不定。

症狀:
(1) 有一大段時間無法閱讀「文字」。

(2) 嚴重退化到無法想像的地步,可能是國中,可能是小學,可能是奶娃。

(3) 開始動手整理房間,可能還會找到寫論文時一直找不到的資料,但是通通來不及了。

(4) 一段時間的狂玩,狂吃,狂睡,除了當豬之外,沒別的志願。

(5) 煩惱畢業即失業的問題。







\section{It’s OK to quit your Ph.D.}

\url{https://www.sciencemag.org/careers/2019/06/it-s-ok-quit-your-phd}
这篇文章主要讲述了一些博士生退学的原因和后来的发展。

刚进入研究生院的时候,他想:我会进入梦想中的生活,我会深入、创新性的思考真正有趣的问题。
但是现实是不同的,他感觉自己被压缩进一个越来越小越来越小越来越小的角落和子领域,他很受伤。

在会议两个月后,他依然无法改变他的感受: 他并没有做任何有冲击力的事情。
于是,他走进导师的办公室,宣布他要退出博士阶段,
他的导师,很支持他的决定,并询问他今后的计划。
他说:我不知道,我只是要做一些不同的事情。

大约1/4的科学和工程博士生在前三年退学。
但是对于每个具体的学生,退学也许是一个更好的选择。
Science采访了九个退学的博士生。

丧失对研究的兴趣

Muredda 前两年做的工作非常奇特,而且和其他人都不同,这让他现得很像个怪人。同时他对研究的好奇心也在退散。、
他不停的强制自己必须要获得学位。他不想让父母伤心。
他坚持了下来,最终倒是同意他完成毕业论文。
但是他的心已经不再学位上了。
他决定继续留在学院,但是在健康护理通信从视一份非学术工作,并且在工作之余撰写论文。、
但是他说,他的激情再也没有回来,而且越来越糟糕。
最后,他在8年博士生涯后退学。

尽管刚开始非常痛苦,但是事情却在好转。
他现在是Harrison的CEO,


追求不同的激情/转行
Toby Hendy, 退学的原因并不是不喜欢研究,而是她希望从事科学交流。
她希望在大学任教。但是自从她开始在YouTube上网上教学,她认为对她而言传统的大学教育方式并不是最佳的。
在YouTube上,她能够接触到更多的人。

Luke Mitchell 厌倦了日复一日的工作,开始寻找更极客的激情。
他转行从视科幻写作。

对学术界的失望

有些前博士生说他们在经历过人文文化后,对学术生涯失去了兴趣,也失去了对博士学位的兴趣。


回顾

许多退学的前博士生都认为这是一场失败。
Martinsek 说,由于学习物理的女生寥寥无几,她很担忧她会让自己的性别蒙羞,这也是让选择退学如此艰难的原因之一。
最终她说服自己摆脱阴影的是:“退学并不意味着我无法成功,也不意味着女性无法成功, 我只是为自己做决定而已”。

Schulz 回忆说每个人都认为他很傻。他不知道应该如何投简历寻找一份正常的工作。

Tye说退学是她做过的最困难的决定,但也是她最自豪的事情。
她认为这是一场解放。她喜欢电影楚门的世界:退一步海阔天空(你意识到这个小世界之外还有更多)。

Muredda, 事后诸葛亮/马后炮, 重新评价她进入研究生院的动机。
“我想,我进入了科学,半对半错”, 她很迷恋生命科学,很好奇事物的工作原理。
但是不管什么原因,那三个字母对我都意味着很多。PhD并不是我进入科学的原因。

