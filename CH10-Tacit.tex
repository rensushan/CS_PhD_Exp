\chapter{隐藏技能}


\section{论文写作方法}

\subsection{教授为什么没有告诉我}

摘录: 写作过程中各种挫折与与收获的转折,真的开始写,什么写作手册都没有效,因为我们很难将那些书中列举的
好论文的构成元素以及写作的步骤,这些正式的特性,运用到我们每个人面对的情境。

总之,论文的研究与写作是一个漫长的旅程,没有捷径可抄小路。所谓“没有灵感”的俗民说法,把作品幻化为作者英明的产出,其实与文字工的现实大相径庭。不论是文艺或学术性的写作,都有规训与苦力的面向,作者必须时时鞭策自己,对抗自己在思考与文字上的疏漏、跳跃与怠惰。


在写作论文或报告时,要练习角色的对调,练习将老师视为学生、将自己当作老师。

评审学生论文经常发现的问题:
1  文献回顾变成家具型录,反无杂乱,而非批判性的回顾。critical review
2  引用资料变成剪贴拼图。应当发展分析性的概念   analytic concepts
3 书目引注充斥疏漏错误

研究是一个资料收集、分析与写作的循环反馈过程。

不要将文献视为权威,而应当将其视为可能一些有用但可能错误的想法。

如果想知道论文的被引用情况, 可使用索引资料库(citation Index, 主要有 Science Citation Index, Social Science Citation Index, Arts and Humanities Citation Index三种 )


2004年 Elsevier公司推出的Scopus资料库,收录期刊丰富,可以查询引用和被引用情况。

书籍:

国家级图书馆
美国国会图书馆: http://catalog.loc.gov, 
英国国家图书馆: http://catalogue.bl.uk

各个大学图书馆
商业网络书店: 如 Barnes \& Nobel(www.barnesandnoble.com)   ,亚马逊,
英国TSO Online Bookshop ( www.iso.co.uk/bookshop/bookstore.asp)

补充: 计算机类:

外加book review, 

博硕论文

网站:  www.altavista.com

搜索引擎:  Google常用双引号把关键词圈起来,表示这几个字必须连在一起

scholar.google.com

询问专家: 

图书馆:

美国国会图书馆分类, LCC
英国杜威十进制分类系统:DDC

Endnote: 
可以把MEDLINE, PsysINFO等数据库文献直接导入Endnote


使用文献,而不是Show文献
(这一章节很精彩 )


何春蕤(1997)提出批判阅读的重要性。
所阅读的文献一定超过你所需要的,但是文献回顾不是展示你如何博学的场所。
文献回顾,不应只是条列式叙述别人说过什么,做过什么。必须要有自己的观点,
要对文献进行批评,要有整合文献的功夫。
理论回顾也必须与资料分析前后呼应。

Flowerdew提出文献回顾的5C原则: 
Comprehensive, concise, coherent, cumulative, critical.
全面,  精简, 一致, 累积, 批判。

文献回顾要能说服读者,就逻辑上来讲, 下一步应该就是你的这个研究。
就算不是,也要说明你的研究如何与先前研究衔接、解决什么过去尚未解决的问题、你又增加了什么新的观点。

文献结构有许多不同的方式,重要的是要说明与你研究的相关性。研究者可以把相同观点以及相互竞争的观点分别讨论、可以叙述理论发展的历程(和线性时间不一定吻合)、可以按照主题分类、也可以将之分为理论/经验/方法三个部分,或者从比较一般到较特定的理论来加以组织。最重要的是要对文献加以批判,读者想要知道你对文献的看法、文献的长短处、观念是否有突破之处、对既有的知识基础增加了什么,然后可以把你的研究放置在既有知识基础的某个位置。文献回顾要能说服读者,就逻辑上来讲,下一步应该就是你的这个研究。就算不是,也要说明你的研究如何与先前研究衔接、解决什么过去尚未解决的问题、你又增加了什么新的观点。(P67)


何春蕤指出学位论文最常出现的毛病就是,文献回顾这一章看不到作者的声音,
它可能是任何一个人所写的。
既没有个人的观点,也没有对文献加以批评和运用。
文献回顾应该说明自己在学术地图上准备占有的位置,指出我们距他们有多远、
相对于他们有什么差异,否则充其量就是鹦鹉而已。

切记原则有三:
1 能够回到原典本身,就阅读原典,除非语言不通或找不到原典
2 记得将概念提出的贡献归给原创者,也就是说即使你没有阅读原典,但在正文中可以说明是谁首先提出这个观点的
3 要让读者知道你究竟阅读的是拿一篇文章,读者自然也就有足够资料能够回溯到最前面原始的文章

论文的核心: 发现与分析。

建立理论需要的不是天才,而是方法。

尽量不使用具有种族、年龄、性别、能力歧视的用语,已是学术界的共识。

使用不同的字来代替, 例如用police officer代替policeman,  
people代替mankind, 用representive 取代congressman,
用human resource或workforce代替manpower。

如贵希望精炼自己的中文,免受西式中文的荼毒,可以参阅这两本书: 思果(1972)《翻译研究》,台北:大地;
思果(1982)《翻译新究》,台北:大地,


写实主义故事关心所知(the known),却不知他如何而知(knowing);
而自白的故事则关心研究者(knower),却常忽略了为何。
印象式故事呈现田野工作(the doing of fieldwork),而不只是作者(the doer)
或成果(the done)。
它同时讨论文化以及研究者知的方式,企图同时观看subject和object,
来连接the knower与the known。

论文写作症候群:P134(大段引用)

首先要谨记在心的是,写作并非成竹在胸,把已经在心中想好的、很完整的东西,如实地再现;相反地,写作本身就是一种思考与分析。只有真的下笔写出来,才真的了解自己到底懂得多少、其间是否有破绽、有没有矛盾不清之处。

因此,绝对不要等到分析架构都很完整的时候才开始动笔写。否则觉得这样写好像不对,那样写好像不够完整,又怕将来万一分析架构修改了,会不会以前写完的就白写了。如果有这样的顾虑,其结果往往是困在那里,迟迟无法动笔。

Wolcott(2001)就说,写作永远不会嫌太早。他又引用Warren的话,认为写作是一项自负(arrogant)的行为,因为你明知自己知道的不够多,可是还是要动笔写作。

所以不要总是以“等我读完架上这几本书”、“等我把概念之间的关系都厘清楚了”为理由,而停滞在那。
只要有一点想法,就动笔写下来,反正还可以持续修改。

最好的方式是,在研究过程当中,就不停地记笔记。……这些笔记很容易就成为将来论文正文的基础。开始写的时候,尽量就一次写的详细些,将来删减总是比增添容易。

写论文如果卡住了,问题通常不在于写作,而是概念上的问题,亦即不知如何分析。面临这种状况,除了两眼凝视空荡荡的墙壁之外,可以做几件事(参考Wolcott,2001)。

一是邀请你的同学或学长姐喝杯下午茶(记住不要吃大餐,否则忙着剥虾切肉、赞叹食物的美味,就没有时间说话讨论了),他们也许可以给些你意想不到的见解。即使没有,你也可能在对谈的过程中,更能抓住论文的重点。

另外一种方法就是持续写作。不要把写作当作写不下去的问题来源,而是当作解决问题的方法。记住用纸笔来思考(think on paper)(参考Wolcott,2001, P39 ), 而不是在脑中空想。正如Becker所说的身体铭刻(physical embodiment)。即使没有新的想法,也可以把脑中的混乱、待解决的问题写下来。

还有就是论文写到一定程度,每天与论文稿面面相觑,会产生冷感,不只是看不到错字,不知道自己惯用的词语,也看不到写作分析的局限。这个时候,除了与好友同侪(chai)分享讨论之外,还可以刻意营造论文稿的陌生感。例如大声的“念”论文,声音和文字的感觉不同;从论文的后面往前读;更动排版格式(边界、字型、字元间距等),让文字所处位置改变,以去除阅读惯性,让你对论文有新鲜感;尝试在每一节中删除一段、每一段删除一句、每一句中删除一个字,让文字更为精简;偶而换个陌生空间来阅读与写作。

论文写作是长期抗战,需要为自己营造一个舒适的写作环境,包括时间和空间。

论文有些部分好写(像是剪辑访谈引文、整理参考书目、描述研究方法、把计划书的文献问题改写、可以一下就写好几页),有的部分比较难写(像是资料分析、结论、有时可能坐在书桌前几个小时就是写不出几个字),可以交叉运用,不要先把好写的写完,然后每天一起床就要面对最困难的部分。


作者江源慎指出(原文参考:\ref{phd_diagnostic} ), 研究生准备开始写论文的时候,经常觉得他人的研究成果不过尔尔,狠狠将之痛批一顿;同时觉得自己天纵英才,必将写出瞩目佳作。开始动笔写论文之后,发现自己懂得东西太少,但是已经来不及改题目;开始思考从来不曾思考过的问题,例如“我活着是要干嘛”。

2 其实真正的意思是...

3 研究生的症状

寫論文時,鍵盤與手指頭出現過“同極互斥”的現象,但打B或玩GAME時此現象便消失;一星期有3天以上,下午兩點以後才吃第一頓飯;對論文以外的一切事物都感到興趣;對英文F開頭以及S開頭的字感到特別親切;對自己的能力以及適不適合走學術這條路的問題有過前所未有的深刻體會及考慮。

5 写作论文是一个漫长的旅程

在这篇BBS文章中,在美国留学的地理学博士讨论四个撰写论文过程中的议题:有个好的互助团体,找个好的editor可以减轻英文写作压力,与指导教授之间的关系,寻找减轻压力的方法。
他认为人在说明自己论文的时候,其实脑筋正在高速运转而不自觉,有的时候就在说的过程中,
把答案想出来或者把论文组织的更好,所以需哟啊找几个具有敏锐洞察力又愿意倾听的朋友(相同与不相同学术领域)来当听众。

写论文的过程,充满了孤单寂寞与焦虑,需要心理调适、运动、打坐、做菜等都可以。有的人专研中西养生汤与花草秘方、枸杞、红枣、当归、玫瑰、茉莉、菊花天天换着喝;有的人天天散步,硬是走出一条哲学家小径。

结论并不是把研究发现章节中的东西又重复说一遍,而是要将研究结果再进一步理论化。
有可能、有必要的时候,再对政策或规划提出建议。
重点是“结论不是摘要”,有的论文甚至在每一章的每一节都有一个小结,
每一张结束时再一个小结,最后再一个结论,
等于是重要的论点,多的时候可以写上三四次。 
再精彩的论点,出现多次之后,读者读了也会厌烦。

结论是你现在把研究做完了,静下心、退一步、跳高一层,想想你的研究整体的意义在哪里、学术社群
为什么需要多一篇你的论文、把你论文放置在相关研究领域的版图里。

有的论文有如下的毛病:章节各自独立、缺乏整体的连结。
文献回顾列举许多理论,可是写完这章,理论就消失不见了。
经验研究结果和文献看不出有什么关联。,
结论与建议又是另一套东西,
不是从自己的研究结果推到而来,而是抄自其他文献。

论文的建议应该建立在自己经验研究基础之上,不要说大话空话。
不要提出没有证据支持的建议。

未来的研究方向,应该建立在论文的研究基础之上,
进而发问,也就是要问出别人没有做研究所问不出来的问题。

最好是当你进行完这个研究之后,你有新的发现、新的观点与发问方式,
站在你的研究的肩膀上,启发新的研究领域,而不是机械式的推论。
这样论文写完了,就不会是一个结束,划上一个据点,而是开启了另一个研究的起点。


推荐的网上书店:

https://labyrinthbooks.com/
看上去偏重于人文 社科类。

https://www.strandbooks.com/
特色在于可以买到Review books.


\section{给研究生的学术建议 }

作者:  Gordon Rugg, Marian Petre

he Unwritten Rules of PhD Research

北大高等教育文库·学术道德与学术规范系列读本

北京大学出版社


从理论上来讲,博士项目是一项由你自己做主的工程。

可迁移性技能 transferable skills 在学术体制中尤其受欢迎。

经验丰富的研究者通常有一份不断晚上的文献目录,大约五十页到一百页左右。
这份文献目录集里集中了他们所在专业的核心参考文献。

做一份有注解的文献目录。


学术体制:RA(研究助理),Postdoctoral research assistant(博士后研究助理),
lecturer(讲师), Senior Lecturer(高级讲师), Principle Lecturer(主要讲师),
Reader(普通教授),Proper research fellow(资深研究员),  Professor(资深教授), 

滥竽充数者: 
滥竽充数者除了戴棒球帽之外, 还会伴有闲散等坏习惯,这类人认为研究就是浪费时间,
因此倾向于去因特网上搜索,而不去做适当的文献调查;
更令人吃惊的是, 他们非常喜欢设计蹩脚的调查,其中往往夹杂一些惨不忍睹的问卷,
或者一些完全没有现场记录的所谓采访稿件。
这种论文通常以一些众人皆知的简单的常识开始,不是关于因特网的普及, 
就是关于他们的研究领域的任何一个热门话题,
然后就是极尽修剪缩短之能事, 随后“写出啦”的论文充斥着拼写错误和漏洞百出的语法,
另外还附加了彩色饼状图标,目的是为了给评审人留下深刻的印象,
但是这样做会适得其反。

校外评审人可能从来没有见过你,所以评审时着重看论文本身,而不去看你是个多么善良的人
或者你多么用功。 你的论文最好看起来像是专业人士的作品,注意重要的细节(你引用的文章是不是完整?是不是正确?)。
你需要用专业语言写作,就像写给另外一个专业人士一样,避免用教材或通俗文摘中所用的简单语言。

在研究生中流传着许多传言,其中有一两个可能是真实的,但是大部分至多属于传言,甚至有些会产生危险的误导。

教科书适合于本科生,其内容都是很简单的叙述。
作为研究生你应该阅读真正的书籍。
你一般会在期刊文章中或专业书籍中才能看到这样复杂的文字。
在大多数学科中,书籍只能提供最前沿最时尚的信息: 书籍是非常有用的参考资源,
尤其是那些曾经改变某个领域的经典书籍。但这些书并不是唾手可得的,
需要通过寻找得到。

发表论文没有人们想象的那么困难,关键是你必须知道自己在做什么。

因特网很受滥竽充数者的喜欢。 网络可能提供很多不可靠的非专业的消息源。

习得无助感(learned helplessness)

没有变化,你就不会进步。 你经历的任何变化都不是一帆风顺没有痛苦的。
学会接受痛苦,这样你的生活才会发生变化。


有经验的审阅人浏览文章和查找错误的速度是你无法想象的。

持久稳定的阅读。 读书的真谛不在于一口气念很多书,而是在于
始终保持一定的阅读量。 
在这方面,乌龟策略胜过兔子策略。

论文类型: 数据导向型论文(data-driven papers), 

P121:
如果你的文章还没有准备好,就不要拿出来给别人看;应该回去改进文章,而不是把自己不够格的东西展示出来再为此道歉。

要显示出你是专业人士,那么使用领域内专业人士的语言和写作惯例。
如果你不清楚,就应该学习。
要使你的语言和其他非语言的方面都与没有经验的起步者区别开来。

要客气地、不带挑衅的传达这个信息(重要的东西)。不能与你的读者产生对抗,但是应该
让你的读者感觉到,你是一个专业人士,你不会给他们看二流的东西,浪费他们的时间。
一个有用的诀窍就是要换个角度考虑问题,不要让读者感觉到你是在说:“看着我”,而是要让读者
感觉到你是在说:“看这个”。

文献和研究方法一样,应该是你的朋友,你掌握的文献越多,你能自信地应付的问题也就越多。

永远都不要表现出自己的弱点,向读者道歉,请求读者的原谅。
如果你的论文写得很好,你就没有必要道歉或者乞求同情。
如果你的论文没写好,那就不要把它拿出来见人--回去把它改好了再说。

有很多词语和句子都会表现出你的软弱。其中有一类成为“推脱话”(weasel words),这类话
的目的是帮助你躲避必须做明确判断或采取明确行动的局面。
推脱话再学术写作中不应该出现,在学位论文中更不应该出现。
比如说某件事“好像(seems )”另一件事,那是不够的。
到底是还是不是?
推脱话说明作者思考的不够深入,或是在做无根据的推断。
(模棱两可的话)

如果你对某个问题不了解,那就花时间和精力去学习,直到你理解为止。

博士论文呗广泛认为是最高形式的学术写作,它对内容、准确性、
论证以及对全文的统筹规划方面的一号求都要高于对单独发表的学术文章的要求。
它是一部“杰作”,这并不代表它是“完美作品”,而是意味着
它包含的写作技巧、技能、形式和工能可以站是一个学生的全部才华。


P141
清晰、简单、权威、诚实,这几点是作者语言的正面特性。


些什么都比不写好。不要担心它不够好。
写论文的过程就是不断把它变得足够好的过程。


P156 
在内容方面容易犯得经典错误很多,其中之一就是错误判断你在规定时间内能讲的内容。


陈述报告

在形式方面容易犯的典型错误

事先熟悉一下你将使用的设备,尽可能掌握多种设备,另外要做好设备出故障的准备,拥有后备方案。

新手往往一开始大喊,几分钟后就变成了喃喃自语。

你需要让观众感受到,你是个专业人士,你对相关课题有透彻的掌握。

会议是观察不同陈述报告风格的极佳场所。
做别人陈述报告的听众是一个获取陈述报告经验的很好的途径。你可以观察别人是怎么做陈述报告的。
你可以利用这个时间留意陈述这使用了哪些内行人的技巧,或者犯了什么错误,
这样你就能简介改善自己的陈述报告风格。
包括如何在别人提问前就积极地回避对自己不利的问题。

看到别人因为一个幼稚的问题或幼稚的发言而被攻击是很痛苦的,
但要比亲身经历这类情形好受多了。
你也可以在这种场合了解到你的同辈研究员对于某些行为的看法。

无知往往表现为做前人已经做过的研究(在图书馆呆一天可以节省六个月做多余研究的时间)。

尼尔斯 波尔(Nils Bohr)曾指出:"科学不是那些有趣的东西,而是那些奇怪的东西“。伟大的研究往往是意想不到的。最糟糕的研究室没有带给你任何新信息的研究;你的失败提供了什么信息

学会蟑螂原则(the  cockroach principle). 它们一条关键生存原则是找到一个舒适、安全、用于藏身的洞,
这样只要一有危险,它们就可以钻进洞里。这对你的启示是, 
你应该为自己寻找一个保护人或避难所。
理想情况下,你的保护人和避难所应该分别是你的老板和你的工作。

如果你决定在做研究的基础上开展你的事业,那么你就有必要设定一个综合计划。
随便抓住一个机会,做一些最基本最简单的研究工作是很轻松的,
也是很有诱惑力的行事方式。
但你必须抗拒它们的诱惑力。
它们会浪费你的时间,而这些时间可以用于做其他更有意义的事。

专家们在逻辑和抽象推理方面并不比新手具有更多又是,许多其他方面则与新手水平相当。
他们明显强于新手的是,他们善于组织对自己研究领域的专业知识,
他们掌握的事实也比新手多得多。
另外,他们比新手更善于制定研究战略。

一个将知识组织起来的有效方法是作规划。

答辩: 蘑菇的故事,啄木鸟的故事

P201:评审人的问题

让你证明自己知识的证实性问题; 深度证实性问题(让你证明你的知识不仅仅限于表面);
学术性问题(让你证明你对自己的领域和研究都很熟悉);挽救性问题;
递进式问题;

杀手问题和应对:

评审人期待的就是学生对自己的错误和障碍做出充满智慧的反应,并且从中学到东西。

一般的答辩问题:P204

学会耐心的听是一种难以估价的研究技能。

现实更理性,成功的研究者被两个或者三个需要所驱动:身份、权力和满足好奇心。



