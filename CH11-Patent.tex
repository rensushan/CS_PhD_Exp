\chapter{专利}


\section{申请专利}


\section{查询专利}


专利信息检索相关网址


国内网址:
1、 国家知识产权局 http://www.sipo.gov.cn
2、 中国专利信息网 http://www.patent.com.cn
3、 专利信息中心 http://www.cpo.cn.net
4、 知识产权培训中心 http://www.ciptc.org.cn
5、 知识产权出版社 http://www.cnipr.com

国外网址:
1、 世界知识产权组织 http://www.ompi.int
2、 欧洲专利局相关网站 http://www.ep.espacenet.com
3、 美国专利商标局 http://www.uspto.gov
4、 日本特许厅 http://www.jpo.go.jp
5、 非洲知识产权组织 http://www.oapi.cm
6、 在线信息服务系统网 http://www.dialog.com
7、 科技信息网 http://www.stn.com

\section{专利相关利益}
% http://kyybgxx.cic.tsinghua.edu.cn/kybg/detail.jsp?seq=1974&boardid=220602&pageno=1
来源为: 清华大学关于加强专利工作的若干意见

清华大学关于加强专利工作的若干意见
--经2001~2002学年度第4次校务会议讨论通过--

(2001年11月7日)


拥有大量知识产权,尤其是发明专利,是研究型大学的重要特点。为加强我校专利工作,促进技术创新,特提出如下意见:
  1、发明专利是科技创新的重要体现,在对教师的绩效评价中,发明专利与高水平论文同等对待。专利的拥有量将作为理工医学等院系科技成绩的重要考核指标,理工医学等院系应根据实际情况制定近期应达到的目标及相应的奖励措施。
  2、为鼓励职务发明创造,学校拨专款并积极从政府、企业和个人等方面多渠道筹集资金设立专利基金,用于支持发明专利的部分代理、申请和维持等费用,其余部分由发明人从相关科研经费中支付。
  3、学校对每项授权发明专利给予2000元奖励,每项授权实用新型和外观设计专利给予500元奖励。为鼓励专利技术的转化实施,合同约定的专利实施许可费在汇入学校帐户后,其中20%可作为奖金奖励给发明人。学校对实施专利技术后企业年增利税500万元的专利技术授予“专利金奖”。
  4、为加强专利管理和服务,科技处要按照管理与代理相结合、校内与校外相结合、国内与国外相结合的原则,建设一支高水平的专利管理队伍和以兼职为主的专利代理队伍,学校将给予相应的灵活机制和优惠政策。
  5、本规定自2001年开始实行,解释权归科技处。
  
  