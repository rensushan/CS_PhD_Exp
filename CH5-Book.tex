\chapter{必读之书}


\section{博士专用}

注: 本书单多数来自《给研究生的学术建议》附录。
文本编辑经典:
Copy-Editing:

提高写作能力:
BUGS in Writing,

效率写作:

The Complete Plain Words,

非批评性、细致而精确记录的范例:

The Histories. 

www.cs.berkeley.edu/%7Epattrsn/talks/BadCareer.pdf

https://people.eecs.berkeley.edu/~pattrsn/talks/BadCareer.pdf


\section{个人推荐书单}


\subsection{方法论}

\subsubsection{人生哲学}

品格之路

人生十二法则

临终前最后悔的五件事    邦尼.韦尔

换个活法:临终前会后悔的25件事
作者是大津秀一,2010年3月由中信出版社出版

\subsubsection{学习方法}

教授为什么没有告诉我

如何学习

如何读一本书

刻意练习:如何从新手到大师

思考:快与慢

学会写作: 自我进阶的高效方法

Vera Johnson-Steiner的书《Notebooks of the Mind

\subsubsection{健康和休息}
斯坦福睡眠法

囚徒健身


\subsubsection{效率和习惯}

掌控习惯

态度,  格局,见识 --  吴军


\subsubsection{团队写作和领导}

成为领导者

要领:斯坦福大学校长的十要素

\subsection{科普}

人类简史

今日简史

未来简史

从零到一

文明之光


\subsection{历史}

中国近代史(徐中约,港版)

史记

三国志


汉书 

后汉书

中国近代史  (蒋廷黻)

中国大历史

万历十五年

中国通史 (傅乐成)

中国通史  ( )

\subsection{文学艺术}

1984   必读!

艺术的故事  贡布里希

儒林外史

\subsection{人文}

人类简史

时间简史

南渡北归

临终前最后悔的五件事情

%菊与刀

\subsection{传记}

宋徽宗


曾文正公家书

阿道夫·希特勒(1889—1945):《我的奋斗》

毛泽东传 ?

蒋介石传记 ?

蒋经国传?

邓小平时代

乔布斯

Elon Musk


\subsection{哲学、社会、军政}

亚当·斯密 (1723~1790):《国富论》

君主论

战争论

阿弗雷德·马汉(1840—1914):《海权论》
亚当·斯密 (1723~1790):《国富论》
马尔萨斯 (1766-1834):《人口论》


\subsection{非主流必读 }

墓碑

\subsection{休闲读物}

原则

规模:复杂世界的简单法则

阿道夫·希特勒(1889—1945):《我的奋斗》

\section{计算机和电子工程专业书籍}

\subsection{体系结构}

计算机体系结构量化分析


\subsection{数字电路}


\subsection{人工智能}

Deep Learning

统计学习


\subsection{算法}

TAOCP 

算法导论

\subsection{图计算}


\subsection{高性能计算}


\subsection{操作系统 }


\subsection{编译原理}


\subsection{}

\subsection{IT科普和伦理}

深度学习革命

为什么  The Book of Why

图灵奖得主演讲

终极算法:机器学习和人工智能如何重塑世界

Rebooting AI     Macus

数字化生存/being digital

《智能缔造者:人工智能开发者口述真相》

Architects of Intelligence: The Truth About AI from the People Building It  马丁·福特(Martin Ford)新书,

%大数据时代

\subsection{编程基础}

C++

Python

Python Machine Learning

Deep Learning with Python


\section{图灵奖得主汉尼斯的图书馆 }
注: 本段节选自 《要领:“斯坦福校长领导十得”》, 英文书名为
“Leading Matters”

领导力的本质

华盛顿和他所生活的时代

• David Hackett Fischer, Washington’s Crossing (New York: OxfordUniversity Press, 2004)

这本书讲述了美国独立战争期间,华盛顿在纽约经历了一系列战役失败后并没有退缩,而是继续组织进攻的历史故事。这里所谓的Crossing(十字路口)是一语双关,一方面指代华盛顿的个人转变,他在战败后转而采用了一个新的策略;同时也指代他在特拉华州面对的真正的十字路口。戴维·麦卡洛(David McCullough)的书《1776》也讲述了同一时期的故事。以下的书都强调了华盛顿谦逊的品质和他对平等、贤能政治的看法。• 罗恩·切诺,国家的选择.钱峰,译.北京:北京联合出版有限责任公司,2014.


• 罗恩·切诺,国家的选择.钱峰,译.北京:北京联合出版有限责任公司,2014.

• David McCullough, 1776 (New York: Simon \& Schuster, 2005)

林肯和他所生活的时代

• Doris Kearns Goodwin, Team of Rivers: The Political Genius of AbrahamLincoln (New York: Simon \& Schuster, 2006)

在这个主题下,我不得不从很多对我有极大影响的书中选出几本来介绍给大家。我选择这本书是因为它深刻地讨论了团队建设、合作、谦逊、道德感和勇气等主题。

• David Herbert Donald, Lincoln (New York: Simon \& Schuster, 1996)

• James McPherson, Tried by War: Abraham Lincoln as Commander inChief (New York: Penguin Press, 2008)• William Lee Miller, Lincoln’s Virtues: An Ethical Biography(New York:Vintage, 2003)

• Ronald C. White Jr., Lincoln’s Greatest Speech: The Second Inaugural(New York: Simon \& Schuster, 2002)


罗斯福和他所生活的时代

• David Kennedy, Freedom from Fear: The American People in Depression and War, 1929—1945 (New York: Oxford University Press, 1999)

这是一本历史书而非传记。不过这段历史时期的各个场景都充斥着富兰克林·罗斯福的身影,从他1932年赢得总统选举到1945年去世。这本书的内容包括他在“大萧条时期”的炉边谈话、他为对抗经济萧条和失业潮所采取的措施、他与丘吉尔的关系、他与英国的结盟,也包括他如何动员巨大力量去赢取战争。在这边书中,我们能看到,罗斯福是一个很有决心的领导者。

• H. W. Brands, Traitor to His Class: The Privileged Life and RadicalPresidency of Franklin Delano Roosevelt (New York: Anchor Books, 2008)

• Doris Kearns Goodwin, No Ordinary Time: Franklin and Eleanor Roosevelt—The Home Front in World War II (New York: Simon \& Schuster,1994)

• Jon Meacham, Franklin and Winston: A Portrait of a Friendship (NewYork: Random House, 2003)

其他美国总统及其生活的时代

• Edmund Morris, The Rise of Theodore Roosevelt (New York: ModernLibrary, 2001), Theodore Rex (New York: Modern Library, 2002), ColonelRoosevelt (New York: Random House, 2010)

西奥多·罗斯福是一个很不平凡的人,他有很多身份,他是运动员、知识分子、历史学家、探险者、改革家、农场经营家,同时也是一位了不起的美国总统。他克服了疾病、改革了美国的公共服务系统、打击了垄断财团、创立了国家公园系统、助力了日俄战争的结束,同时,他还在60多岁时探索了亚马孙河的无人区域。这可谓一段传奇的人生。其中,《领袖的崛起:西奥多·罗斯福》简体中文版已出版。

• H. W. Brands, Andrew Jackson: His Life and Times (New York: Doubleday,2005)• Robert Caro, Master of the Senate: The Years of Lyndon Johnson (NewYork: Alfred A. Knopf, 2002)

• Timothy Egan, The Big Burn: Teddy Roosevelt and the Fire That SavedAmerica (New York: Mariner Books, 2010)

• Joseph Ellis, American Sphinx: The Character of Thomas Jefferson (NewYork: Alfred A. Knopf, 1997)



• Ulysses S. Grant, The Personal Memoirs of U. S. Grant, 3 volumes(Cambridge, Mass. : The Belknap Press of Harvard University Press, 2017)

• David McCullough, John Adams (New York: Simon \& Schuster, 2001)• David McCullough, Truman (New York: Simon \& Schuster, 1992)

• Jack McLaughlin, Jefferson and Monticello: The Biography of a Builder(New York: Henry Holt, 1988)


美国开国元勋、早期领导人及其时代


• H. W. Brands, The First American: The Life and Times of BenjaminFranklin (New York: Doubleday, 2000)

富兰克林在很多方面都可以被看作美国开国元勋中最杰出的那位。他不仅仅是政治家,也是科学家和作家。他是个博学的人,来自很普通的家庭。他组建的团队是知识社会最好的模板,他发明的玻璃琴也非常精妙。从他所写的短文到年鉴中,我们都能看出他是个多产且富有洞见的作家。作为一个外交家,他可以说是劝服法国加入美国独立战争的头号人物,而约克镇战役正是在法国海军的帮助下获得胜利的。在他的身上,我们还能找到很多需要赞赏且值得学习的东西。

• 沃尔特·艾萨克森.富兰克林传.孙豫宁,译.北京:中信出版社,2016.


• 罗恩·彻诺.汉密尔顿:美国金融之父.应韶,译.上海:上海远东出版社,2011.

• David Hackett Fischer, Champlain’s Dream (New York: Simon \& Schuster, 2008)

• David Hackett Fischer, Paul Revere’s Ride (New York: Oxford UniversityPress, 1994)


• Jack Rakove, Original Meanings: Politics and Ideas in the Making of theConstitution (New York: Alfred A. Knopf, 1996)

• Cokie Roberts, Ladies of Liberty: The Women Who Shaped Our Nation(New York: HarperCollins, 2016)


美国其他领导者

• David Garrow, Bearing the Cross: Martin Luther King Jr. and theSouthern Christian Leadership Conference (New York: HarperCollins, 1986)

这是个艰难的选择,因为关于这一话题有很多值得一读的书。如果不得不选择一本,这本马丁·路德·金的传记很值得推荐。因为它全面地展现了马丁·路德·金的领导力之旅,从不太情愿的心态开始,到一路遇到的很多挫折。故事的最后,我们会看到一个为领导民众而生的人不顾那些他已经预计到的危险,义无反顾地献身伟大的事业。

• 劳拉·希伦布兰德.坚不可摧.王祖宁,译.重庆:重庆出版社,2015
.

• 菲尔·奈特.鞋狗.毛大庆,译.北京:北京联合出版公司,2016.• 威廉·曼彻斯特.美国的恺撒大帝.黄瑶,译.北京:中信出版集团,2017.

• 罗恩·切诺.洛克菲勒.王鹏,译.北京:国际文化出版公司,2007.

• 凯瑟琳·格雷厄姆.我的一生略小于美国现代史.萧达,译.北京:民主与建设出版社,2018.

• 沃尔特·艾萨克森.基辛格:大国博弈的背后.刘汉生,等译.北京:国际文化出版公司,2012.

• T.J.斯泰尔斯.第一大亨.粟志敏,栗之敦,莫崇晟,译.杭州:浙江人民出版社,2013.


• Sara Josephine Baker, Fighting for Life (New York: New York Review,2013[1939])

• Kai Bird, Martin J. Sherwin, American Prometheus: The Triumph andTragedy of J. Robert Oppenheimer (New York: Alfred J. Knopf, 2005)


• Elisabeth Bumiller, Condoleezza Rice: An American Life: A Biography(New York: Alfred J. Knopf, 2005)• Robert Caro, The Power Broker:Robert Moses and the Fall of New York(New York: Alfred A. Knopf, 1974)


• Lynne Olsen, Citizens of London: The Americans Who Stood with Britainin Its Darkest, Finest Hour (New York: Random House, 2010)


• Condoleezza Rice, Extraordinary, Ordinary People (New York: ThreeRivers Press, 2011)


• William T. Sherman, Memoirs of General W. T. Sherman (New York:Three Rivers Press, 2011)

• Booker T. Washington, Up from Slavery: An Autobiography (variouseditions; first published New York: Doubleday, 1901)


世界其他领导人:古代

• Donald Kagan, Pericles of Athens and the Birth of Democracy (New York:Free Press, 1991)

读完卡甘关于伯罗奔尼撒战争的那本历史书后,我阅读了这本书。书中讲述的伯里克利是雅典城邦黄金时期的绝对领导人。在这30年里,雅典的民主得到扩大,经济力量和地区影响力增加,艺术也繁荣了起来。一些大型项目,比如帕特农神庙也在这一时期开始修建。

• Anthony Everitt, Augustus: The Life of Rome’s First Emperor (New York:Random House, 2006)

• Harold Lamb, Alexander of Macedon (various editions; first publishedNew York: Doubleday, 1946)

• Harold Lamb, Hannibal: One Man Against Rome (various editions; firstpublished 1958)

• Richard Winston, Charlemagne (various editions; first published London:Eyre \& Spottiswoode, 1956)


世界其他领导人:现代

• Robert K. Massie, Peter the Great: His Life and World (New York: AlfredA. Knopf, 1980)

这本书讲述了彼得大帝如何带领俄罗斯从落后的中世纪国家变成一个欧洲强国。他深入欧洲腹地,通过游学和担任造船学徒向别的国家大胆学习。他不顾各方反对,决心要将俄罗斯带入现代社会;身居万人之上,他仍旧保持谦逊,勇于向他人寻求帮助。这些都很可贵。

• 纳尔逊·曼德拉.漫漫自由路.谭振学,译.桂林:广西师范大学出版社,2014.

• 莫·卡·甘地.甘地自传.启蒙编译所,译.上海:上海社会科学院出版社,2015.

• Roy Jenkins, Churchill: A Biography (New York: Macmillan, 2001)• Robert K. Massie, Catherine the Great: Portrait of a Woman (New York:Random House, 2011)• Andrew Roberts, Napoleon: A Life(New York: Penguin, 2014, 2015)


领导者和他们的探险

• Alfred Lansing, Endurance: Shackleton’s Incredible Voyage (various editions; first published 1959)

欧内斯特·沙克尔顿的艰险旅途可谓是最扣人心弦的领导力故事。在被困于南极极地冰冻后,他们的行船沉没了。沙克尔顿带领着船上的人跨越了两个大洋、用救生艇在开放海域航行了1000余英里。沙克尔顿的领导力和团队建设能力在这次探险之旅中起到了至关重要的作用。这些能力帮助他完成了对全体船员的救援工作。

• T. E.劳伦斯.智慧七柱.蔡悯生,译.上海:上海文艺出版社,2016.

• Daniel James Brown, The Boys in the Boat: Nine Americans and TheirEpic Quest for Gold at the 1936 Berlin Olympics (New York: Penguin, 2014)• Maurice Herzog, Annapurna: The First Conquest of an 8,000-Meter Peak(New York: Lyons Press, 1997 [1952])

• Nathaniel Philbrick, In the Heart of the Sea: The Tragedy of theWhaleship Essex (New York: Viking Penguin, 2000)


企业、政府和学术界领导力

• John W. Gardner, Living, Leading, and the American Dream (SanFrancisco: Jossey-Bass, 2003)

加德纳在政府、非营利性机构和学术界都是很成功的领导者。他曾提道:“我们都面对着一系列的好机会,只不过这些机会都隐藏在一些看似无法解决的问题中。”这句话一直都在启发着我。在林登·贝恩斯·约翰逊时期,加德纳曾以共和党身份担任美国健康、教育和福利大臣。他是“国家老年人医疗保险制度”的首席设计师。他恪守自己的原则,以退出内阁的方式表达对美国发动越南战争的反对。他创立了公民组织“共同事业”(Common Cause),也主导了美国公共广播公司的创立。在他离世前,我曾在一个小型午餐会上碰到过他,我永远也不会忘记那次碰面。加德纳的书都是从他丰富的领导力经验中总结出来的。

• 沃伦·本尼斯.成为领导者(纪念版).徐中,姜文波,译.杭州:浙江人民出版社,2016.

• 史蒂芬·柯维.高效能人士的七个习惯(30周年纪念版).高新勇,王亦兵,葛雪蕾,译.北京:中国青年出版社,2018.

• 罗伯特·盖茨.新领导者的破局法则.杨具荣,路玲,译.北京:金城出版社,2018.


• 比尔·乔治,彼得·西蒙斯.真北.刘祥亚,译.广州:广东经济出版社,2008.

• 罗伯特·K.格林利夫.仆人式领导.徐放,齐桂萍,译.南昌:江南人民出版社,2008.

• William G. Bowen, ed. Kevin M. Guthrie, Ever the Leader: SelectedWritings 1995—2016 (Princeton, NJ: Princeton University Press, 2018)

• Kevin Cashman, Leadership from the Inside Out, Becoming a Leader forLife (3rd ed., Oakland: Berrett-Koehler, 2017)

• Gerhard Casper, The Winds of Freedom: Addressing Challenges to theUniversity (New Haven, CT: Yale University Press, 2014)

• Vartan Gregorian, The Road to Home: My Life and Times (New York:Simon \& Schuster, 2003)


从历史中洞见未来

美国历史:19世纪

• Daniel Walker Howe, What Hath God Wrought: The Transformation ofAmerica, 1815—1848 (Oxford, UK; New York: Oxford University Press, 2007)

我是牛津大学出版社出版的美国历史系列丛书的忠实读者,这个系列中的很多书都会在此处出现。丹尼尔·沃克·豪的这本书主要讲述了从安德鲁·杰克逊总统崛起到美墨战争期间的美国历史。这是一段讲述美国高速发展和更加多元化的故事,也提到了宗教对社会发展的影响、社会对于奴隶制和女性权利的观点的分裂,以及美国与墨西哥之间的战争。


• 托克维尔.论美国的民主.江菲菲,译.北京:北京时代华文书局,2018.

• Stephen Ambrose, Nothing Like It in the World: The Men Who Built theTranscontinental Railroad, 1863—1869 (New York: Simon \& Schuster, 2000)

• James M. McPherson, Battle Cry of Freedom: The Civil War Era (Oxford,UK; New York: Oxford University Press, 1988)


• Louis Menand, The Metaphysical Club: A Story of Ideas in America (NewYork: Farrar, Straus and Giroux, 2001)

• Mark Twain, Life on the Mississippi (various editions; first published1883)

• Richard White, Railroaded: The Transcontinentals and the Making ofModern America (New York: W. W. Norton, 2011)


• Gordon S. Wood, Empire of Liberty: A History of the Early Republic, 1789—1815 (New York: W. W. Norton, 2011)

• Richard Zacks, The Pirate Coast: Thomas Jefferson, The First Marines,and the Secret Mission of 1805 (New York: Hyperion, 2005)



美国历史:20世纪


• David Halberstam, The Coldest Winter: America and the Korean War(New York: Hyperion, 2007)

哈珀斯塔姆以讲述越南战争的历史而为大家熟知。这本《最寒冷的冬天》说明了第二次世界大战后的朝鲜战争为紧接着的美国其他对外战争奠定了失败的基调。在朝鲜战争中,美国犯了很多错误:没有很好地为冬天做出相应的战争准备,以及麦克阿瑟对中国支援能力的错误预估。这些都导致美国在战争中承受大量损失,也让战争最终陷入了僵局。在麦克阿瑟与杜鲁门总统公开争吵后,麦克阿瑟也被开除了。这是美国之后一系列基于道德或者政治考量的对外战争的失败的开始。

• Rick Atkinson, 
An Army at Dawn: The War in North Africa, 1942—1943(New York: Henry Holt, 2002), 
The Day of Battle: The War in Sicily andItaly, 1943—1944 (New York: Henry Holt, 2007), 
The Guns at Last Light, The War in Western Europe, 1944—1945 (New York: Picador, 2013)


• Jonathan R. Cole, The Great American University: Its Rise toPreeminence, Its Indispensable National Role, Why It Must Be Protected(New York: Public Affairs, 2009, 2012)

• David M. Kennedy, The American People in World War II: Freedom fromFear, Part II (New York: Public Affairs, 2009, 2012)

• Richard Rhodes, The Making of the Atomic Bomb (New York:Touchstone, 1988)

• Ted Sorenson, Counselor: A Life on the Edge of History (Norwalk, CT:Easton Press, 2008)

世界历史:古代

• John Hale, Lords of the Sea: The Epic Story of the Athenian Navy and theBirth of Democracy (New York: Viking, 2009)

约翰·黑尔是伟大的作家,也是个伟大的演说者。这本关于雅典历史的书教会我很多东西。雅典是第一个高度依赖贸易的地区,之后的罗马也是。雅典也是一个民主政体,在这里的普通公民都能参与博物馆的管理,并保卫着自己的国土。

• 爱德华·吉本.罗马帝国衰亡史.席代岳,译.杭州:浙江大学出版社,2018.

• 希罗多德.不可不知的波斯战争故事.周莹,淘沙,译.北京:航空工业出版社,2003.

• 唐纳德·卡根.伯罗奔尼撒战争.陆大鹏,译.北京:社会科学文献出版社,2016.

• 修昔底德.伯罗奔尼撒战争史.徐松岩,译.上海:上海人民出版社,2017.

• Barbara Mertz, Temples, Tombs \& Hieroglyphs: A Popular History ofAncient Egypt (New York: Dodd, Mead, 1964)

• Ian Shaw (editor), The Oxford History of Ancient Egypt (Oxford, UK; NewYork: Oxford University Press, 2000)

世界历史:近现代到现代时期

• Barbara Tuchman, A Distant Mirror: The Calamitous 14th Century (NewYork: Ballantine, 1978)

塔奇曼的这本书主要讲述了14世纪的历史,描绘了惨烈的战争、普通人的艰难生活和专横的封建体制。他揭开了骑士精神的表象,展现了一个没有太多优点的阶级分化系统。

• 罗杰·克劳利.财富之城.陆大鹏,张骋,译.北京:社会科学文献出版社,2015.

• 罗杰·克劳利.海洋帝国.陆大鹏,译.北京:社会科学文献出版社,2014.


• 罗杰·克劳利.1453君士坦丁堡之战.陆大鹏,译.北京:社会科学文献出版社,2014.

• Dominic Greene, Three Empires on the Nile: The Victorian Jihad, 1869–1899(New York: Free Press, 2007)

• Timothy E. Gregory, A History of Byzantium (Malden, MA: Blackwell,2005)


世界历史:20世纪

• Margaret MacMillan, Paris 1919: Six Months That Changed the World(Malden, MA: Blackwell, 2005)

1919年巴黎和会的主调绝不是谦逊,而是贪婪。欧洲大国联盟的复仇之心和美国威尔逊总统的低效导致施加给德国的赔偿数额是德国不可能偿还的。这最终给希特勒的上台创造了环境。这本书和塔奇曼关于第二次世界大战起源的书都说明了这些问题已为第二次世界大战埋下伏笔,最后造成了不幸的结果。

• 利雅卡特·艾哈迈德,金融之王.巴曙松,李胜利,等译.北京:中国人民大学出版社,2011.

• 巴巴拉·W.塔奇曼.八月炮火.张岱云,译.上海:上海三联出版社,2018.

• Robert K. Massie, Nicholas and Alexandra: The Classic Account of the Fallof the Romanov Dynasty (New York: Atheneum, 1967)



各个文明的历史及其发展

• Jared Diamond, Guns, Germs, and Steel: The Fates of Human Societies(New York: W. W. Norton, 1999)

戴蒙德的这本《枪炮、病菌与钢铁》提出了一个很有意思的假说:是地理和自然因素,而非文化因素,导致了不同社会间的巨大差异。戴蒙德提供的一些例子很有说服力,但在别的一些例子中,他的这个假说似乎不能被证实。弗格森关于文明发展的书能够提供一个有意思的对比。弗格森提供了一些同样有说服力的例子,说明了不同的文化和法律系统导致了一些社会差异。两个不同的思考路径都提供了很好的(如果不能说是完整的)角度去思考这个主题。《枪炮、病菌与钢铁》简体中文版已出版。

• 凯伦·阿姆斯特朗.神的历史(珍藏版).蔡昌雄,译.海口:海南出版社,2013.

• 尼尔·弗格森.货币崛起.高诚,译.北京:中信出版社,2012.

• 尼尔·弗格森.文明.曾贤明,唐颖华,译.北京:中信出版社,2012.

• 托马斯·弗里德曼.世界是平的.何帆,肖莹莹,郝正非,译.长沙:湖南科学与技术出版社,2015.


• 斯蒂芬·平克.人性中的善良天使.安雯,译.北京:中信出版集团,2019.

• Hilda Hookham, A Short History of China (New York: New AmericanLibrary, 1972)

• Barbara W. Tuchman, The March of Folly: From Troy to Vietnam (NewYork: Alfred A. Knopf, 1984)

• Fareed Zakaria, The Post-American World (New York: W. W. Norton,2008)


创新者的密码关于创新者的书:文艺复兴时期到18世纪

• Walter Isaacson, Leonardo da Vinci (New York: Simon \& Schuster, 2017)

这本《列奥纳多·达·芬奇传》是关于生活在文艺复兴时期的达·芬奇的一生。因为他,我们对文艺复习时期的人的印象都是贯通古今、多才多艺的。艾萨克森的这本书向我们展现了一个受好奇心驱使的发明家、艺术家和科学家的一生。即便达·芬奇的很多作品都是未完成的(这是他的一个习惯),他还是给我们的世界留下了不可磨灭的印记。《列奥纳多·达·芬奇传》简体中文版已出版。

• Ross King, Brunelleschi’s Dome: How a Renaissance Genius ReinventedArchitecture (New York: Bloomsbury, 2000)

• James Reston, Jr., Galileo: A Life (New York: HarperCollins, 1994)

• Dava Sobel, Longitude: The True Story of a Lone Genius Who Solved theGreatest Scientific Problem of His Time (New York: Walker, 1995)


关于创新者的书:19世纪

• Janet Browne, Charles Darwin: Voyaging (Princeton, NJ: PrincetonUniversity Press, 1996)

达尔文是个非凡的人物。虽然他最开始在科学研究的道路上也曾迷茫过,在贝格尔号环球航行中他晕船很严重,但他对科学理论的好奇心、一丝不苟的观察和记录最终帮助他发现了生命最基本的原理之一。

• David McCullough, The Great Bridge: The Epic Story of the Building ofthe Brooklyn Bridge (New York: Simon \& Schuster, 2012)

• David McCullough, The Path Between the Seas: The Creation of thePanama Canal, 1870—1914(New York: Simon \& Schuster, 1977)

• Witold Rybczynski, A Clearing in the Distance: Frederick Law Olmstedand America in the 19th Century (New York: Touchstone, 2000)

• Marc Seifer, Wizard: The Life and Times of Nicola Tesla Biography of aGenius (New York: Citadel, 1998)

• Randall Stross, The Wizard of Menlo Park: How Thomas Alva EdisonInvented the Modern World (New York: Three Rivers Press, 2007)



关于创新者的书:20世纪

• David McCullough,The Wright Brothers (New York: Simon \& Schuster,2015)我特别喜欢麦卡洛写著的关于创新者和创新的书,尤其是这本《莱特兄弟》。莱特兄弟拥有激情、好奇心、不屈的意志和长远的愿景。他们致力于理解飞行的原理,飞行控制问题的解决成了他们获得成功的关键。• 沃尔特·艾萨克森.史蒂夫·乔布斯传.管廷圻,魏群,余倩,赵萌萌,译.北京:中信出版社,2014.• 沃尔特·艾萨克森.创新者.关嘉伟,牛小婧,译.北京:中信出版集团,2017.• 迈克尔·马隆.三位一体.黄亚昌,译.英特尔传奇.杭州:浙江人民出版社,2015.• 莱斯利·柏林.硅谷搅局者.王天,译.成都:四川人民出版社,2019.

• Andrew Hodges, Alan Turing, The Enigma of Intelligence (New York:HarperCollins, 1985)

• Michael S. Malone, Bill \& Dave, How Hewlett and Packard Built theWorld’s Greatest Company (New York: Portfolio, 2007)



最伟大的智力冒险科学、数学和科技:历史及其发展

• Siddhartha Mukherjee, The Emperor of All Maladies: A Biography of Cancer (New York: Scribner, 2010)

在这个板块下有很多不错的书,但穆克吉的《众病之王》十分让人着迷,这是一本关于癌症治疗历史的书,会让读者理解到癌症的本质以及推动医疗进步的困难之处。《众病之王》简体中文版已经出版。

• 比尔·布莱森.万物简史.亚维明,陈邕,译.南宁:接力出版社有限公司,2017.

• 史蒂芬·霍金.时间简史.许明贤,吴忠超,译.长沙:湖南科学技术出版社,2012.

• 侯世达.哥德尔、艾舍尔、巴赫:集异璧之大成.严勇,刘皓明,莫大伟,译.北京:商务印书馆,1996.

• 丹尼尔·卡尼曼.思考,快与慢.胡晓姣,李爱民,何梦莹,译.北京:中信出版社,2012.

• 悉达多·穆克吉.基因传.马向涛,译.北京:中信出版集团,2018.

• 罗伯特·萨波斯,斑马为什么不得胃溃疡.穆志山,译.长沙:湖南科技出版社,2010.

• 纳特·西尔弗.信号与噪声.胡晓姣,张新,朱辰辰,译.北京:中信出版社,2013.

• Manjit Kumar, Quantum: Einstein, Bohr, and the Great Debate About theNature of Reality (New York: W. W. Norton, 2008)

• Leonard Mlodinow, Euclid’s Window: The Story of Geometry fromParallelLines to Hyperspace (New York: Touchstone, 2001)

• Robert Sapolsky, Monkeyluv: And Other Lessons on Our Lives as Animals(New York: Vintage, 2006)

• Leonard Susskind, The Black Hole War: My Battle with Stephen Hawkingto Make the World Safe for Quantum Mechanics (New York: Little, Brown,2008)

• Lewis Thomas, A Long Line of Cells: Collected Essays (n. p. : Book of theMonth Club, 1990)

• Neil deGrasse Tyson, Astrophysics for People in a Hurry (New York: W.W. Norton, 2017)



什么是值得过的人生

精彩的小说

• David Brooks, The Road to Character (New York: Random House, 2015)

这也是个艰难的选择,因为在这个主题下也有很多内容很好且感人的书。我选择这本《品格之路》是因为它讲述了多个领导者的故事:从弗朗西斯·珀金斯到德怀特·艾森豪威尔。那些我认为很重要的领导力特质都能在布鲁克斯书中所写的人物中看到。

• 玛克斯·奥勒留.沉思录.梁实秋,译.南京:译林出版社,2012.

• 安妮·弗兰克.安妮日记.宁瑛,译.北京:商务印书馆,2017.

• 阿图·葛文德.最好的告别.彭小华,译.杭州:浙江人民出版社,2015.

• 保罗·卡接尼什.当呼吸化为空气.何雨珈,译.杭州:浙江文艺出版社,2016.

• 兰迪·波许,杰弗里·托斯洛.最后的演讲.吴笑寒,译.北京:南海出版公司,2018.

• 埃利·威塞尔.黑夜.袁筱一,译.北京:南海出版公司,2018.

• Saint Aurelius Augustinus, Confessions of Saint Augustine (variouseditions; see, for example, London; New York: Penguin, 1961)

• Abraham Verghese, My Own Country: A Doctor’s Story (New York:Simon \& Shuster, 1994)

• Abraham Verghese, The Tennis Partner (New York: HarperCollins, 1998)

• Antheny Doerr, Four Seasons in Rome: On Twins, Insomnia, and theBiggest Funeral in the History of the World (New York: Scribner, 2007)


那些对我有深刻影响的小说家

• 但丁:《神曲》(Divine Comedy),特别是其中的《地狱》(Inferno)一篇。

• 艾萨克·阿西莫夫(Isaac Asimov):

他的基地和机器人系列作品非常有意思,这些小说都讲述了在遥远的未来,这些东西对人类来说意味着什么。

• 简·奥斯汀:通过她的小说,我们能欣赏到作者优美的文字,能够体察到作者自身的个性。除此之外,我们能看到她对人类情感精妙的描写,特别是情感如何影响个体的决定。

• 勃朗特姐妹(The Brontë Sisters):《简·爱》,《呼啸山庄》和《怀尔德菲尔府上的房客》(The Tenant of Wildfell Hall)。

• 威拉·卡瑟(Willa Cather):一些关于美国西部的小说。

• 查尔斯·狄更斯:在他的书中我们能看到精美的语言,能体悟到他自身的性格,能读到他笔下的英国社会黑暗面故事。《双城记》至今是我最爱的小说之一。狄更斯对正义、爱情、坚韧和牺牲精神的讨论会一直流传下去。小说开头的第一句极其精妙。


• 西奥多·德莱塞(Theodore Dreiser):
他的小说展现了很多最终导致悲剧的个人选择。

• 乔治·艾略特(George Elliot):本名玛丽·安·伊文斯(Mary Ann Evans),在她精美的小说中,我们能看到她善于描述人物复杂的性格和丰富的情感。

• 伊丽莎白·盖斯凯尔(Elizabeth Gaskell):她的小说讲述了一系列关于工业革命时期穷人的悲凉生活以及爱情战胜一切的故事。

• 托马斯·哈迪(Thomas Hardy):他的小说描绘了各种各样或正义或邪恶的行为,同时也讲述了邪不压正的故事。

• 弗兰克·赫伯特(Frank Herbert):在他的“沙丘”(Dune)系列中,他用奇幻的方式描绘了一个充满着科技感的奇幻世界,同时他也展现了善、恶、领导力和个人牺牲精神等主题。

• 荷马:《伊利亚特》和《奥德赛》两书讲述了关于道德和伦理选择的一些故事。

• 维克多·雨果:他写了两部描绘邪恶、讲述正义终将胜利的伟大小说。


• 亨利·詹姆斯(Henry James):关于浪漫、骄傲和人类心理学的小说。

• 安·兰德(Ayn Rand):她描述了野心、自由企业和个人所得的重要性以及这些东西的消极影响(我是这么看的)。

• 莎士比亚:他通过喜剧、悲剧和历史剧将人类情感展现得淋漓尽致。• 华莱士·斯蒂格(Wallace Stegner):他不仅写就了很多关于美国西部的小说,同时还成立了“斯坦福大学创意写作项目”。

• 约翰·斯坦贝克(John Steinbeck):他用幽默且感同身受的笔触讲述了很多关于人的性格和各种挑战的故事。

• J.托尔金(J. Tolkien):他的《魔戒》三部曲用极其创新和魔幻的方式讲述了关于正义和邪恶的、意味深长的故事。

• 安东尼·特罗洛普(Anthony Trollope):他的小说,尤其是《巴塞特郡纪事》(Chronicles of Barsetshire),讲述了维多利亚时期英国的性别平等及其他社会问题。

• 马克·吐温:我们能读到他时而幽默、时而悲怆的语言。他可以说是美国最伟大的小说家。







\section{其他的推荐作品}

\subsection{电影}

国产片:

 1942, 霸王别姬,  
 疯狂的石头
 血战台儿庄
 
奥斯卡

 撞车
 阿丽塔,阿凡达
 魔鬼终结者
 万能钥匙
 
其他:
 共同警备区,出租车, 辩护人
 摔跤吧爸爸   
 
 


\subsection{艺术}