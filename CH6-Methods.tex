\chapter{研究方法 }


\section{名家观点}


\subsection{如何在学术领域一事无成}

来源: https://people.eecs.berkeley.edu/~pattrsn/talks/BadCareer.pdf


如何实现糟糕的博士生涯:

集中精力获得高GPA

减少课程数量和品味

不信任导师

不加班

努力早点毕业

不开会

不练习写作或谈话






\subsection{沈向洋:有效的科研法则}
网络录音连接
\url{https://www.youtube.com/watch?v=U6r3R87AKHI&feature=youtu.be }

  一流高手提问题, 二流高手解问题, 三流高手抄问题。
  
 做学问的几种方法:  挖洞法, 枪扎一条线, 棒扫一大片, 吸星大法
 
  复杂问题简单化, keep it simple and stupid
  简单问题复杂化: keep it complex and complicated
  
  
\subsection{MIT人工智能实验室: 如何做研究?}

吴恩达


\subsection{华人生物学家蒲慕明:写给实验室博士的邮件}
著名的华人生物学家蒲慕明先生曾经有一封非常著名的邮件在网上广为流传,这封邮件是蒲先生写给自己实验室所有博士生和博士后的,其中的观点我完全赞同。

这封邮件写得语重心长,从中可以看出蒲先生的良苦用心。我把这封email转给了我实验室的所有学生。

其中的一段是这样说的:我认为最重要的事情就是在实验室里的工作时间,当今一个成功的年轻科学家平均每周要有60小时左右的时间投入到实验室的研究工作......我建议每个人每天至少有6小时的紧张实验操作和2小时以上的与科研直接有关的阅读等。文献和书籍的阅读应该在这些工作时间之外进行。

——施一公

以下为蒲慕明教授写给他实验室博士email全文(英文原文/中文译文):

Lab members:

Over the past several months, it has become clear to me that if there is no drastic change in the lab, Poo lab will soon cease to be a productive, first-rate lab that you chose to join in the first place. Lab progress reports over the past six months have clearly shown the lack of progress in most projects.

one year ago, when we first moved to Berkeley, I expressed clearly to everyone my expectation from eachone in the lab. The most important thing is what I consider to be sufficient amount of time and effort in the lab work. I mentioned that about 60 hr working time per week is what I consider the minimal time an average successful young scientist in these days has to put into the lab work.

There may be a few rare lucky fellows like Florian, who had two Nature papers in his sleeve already, can enjoy life for a while and still get a job offer from Harvard. Nobody else in the lab has Florian's luxury to play around. Thus I am imposing strict rules in the lab from now on:

1. Everyone works at least 50 hr a week in the lab(e.g.8+ hr a day, six days a week).This is by far lower than what I am doing every day and throughout most of my career. You may be smarter or do not want to be successful, but I am not asking you to match my time in the lab.

2. By working, I mean real bench work. This does not include surfing on the computer and sending and receiving e-mails for non-scientific matters unrelated to your work (you can do this after work in the lab or at home, and excessive chatting on nonscientific matters. No long lunch break except special occasions. I suggest that everyone puts in at least 6 hr concentrated bench work and 2+ hr reading and other research-related activity each day. Reading papers and books should be done mostly after work. More time can be spent on reading, literature search and writing during working hours when you are ready for writing a paper.

3. I must be informeded in person by e-mail (even in my absence from the lab)when you are absent from the lab for a whole day or more. Inform me early your vacation plan. Taking more than 20 working days out of one year is the maximum to me. In fact, none of you are reporting any vacation and sick leave on your time sheet (against the university rule, although I have been signing the sheets)), but you know roughly how many days you were not here. On the whole, I understand and acceptthe fact that you may not fulfill the above requirements all the time, due to health reasons, occasional personal business. But if you do not like to follow the rules because it is simply a matter of choice of life style, I respect your choice but suggest you start making plans immediately and leave the lab by the end of January 31. I will do my best to help you to locate a lab to transfer or to find a job. If you do accept the conditions I describe above, I am happy to continue to provide my best support to your work, hopefully more than I have done in the past. I will review the progress of everyone in the lab by the end of June of 2002. I expect everyone to have made sufficient progress in the research so that a good paper is in sight(at least to the level of J.Neuroscience. If you cannot meet this goal at that time, I will have to ask you to prepare to leave my lab by the end of August.

实验室的每一位成员:

在过去的几个月中,我发现如果再不对实验室进行一次剧变,很快,Poo实验室就不是那个你们最初选择加入的高产出,第一流的实验室了。

从过去六个月的实验室进展报告中可以清楚地看出大多数目都没有什么进展。在我们一年前刚搬进伯克利时,我就清楚地表达了我对实验室中每一员的期望。

我认为最重要的一点就是我希望大家保证在实验室中付出足够的时间和努力。

我提到了每周60小时是我所认为的一个成功的年轻科学家每天呆在实验室的时间的最低限。当然,有极少数幸运的同仁如Florian,他已经发表了两篇Nature并有了一份在哈佛的工作,是可以稍稍的享受下生活了。

但是实验室的其他人都不能像Florian这样奢侈的享乐。

从现在起,我在实验室里立下严规:

1.每个人每星期必须在实验室50小时以上(比如说,每天八小时,每周六天。)这比我整个职业生涯每天所花的时间少得多。你们可能会比我聪明或者并不渴望成功,所以我并不以我的标准要求你们。

2.工作,是指真正的搞研究,不包括上网,接发与科研无关的邮件(你可以在工作完成后在实验室或家中做这些)和过度的与科研无关的聊天。除了特殊情况,午餐后的休息时间不要太长。

我建议大家每天花至少六小时在自己的科研工作上,2个多小时做阅读或其他和研究相关的活动上。阅读文献和书籍应放在工作之后。当你打算着手写论文时,建议将更多的工作时间用来阅读,搜索和撰写论文。

3.当你们超过一天不在实验室时,必须用e-mail告知我(即使我不在实验室)。提前告知我你们的假期计划。

对我来说,每年最多可以请20天的假。你们没有一个人在你们的时间表上汇报了你们的假期和病假(尽管我给你们签字了,但这是违反校规的),你们自己知道你们每年大概有多少天是没有待在实验室里的。

总而言之,我知道出于一些个人的临时状况,健康因素等,让你们时刻遵守这些规矩是不现实的。但如果你们出于个人生活方式不同而不遵守这些规定,我尊重你的决定并建议你早作打算,在1月31日以前卷铺益走人,我会尽力帮你转去另一个实验室或找份工作。

如果你能接收我上面说的这些条件,我会非常开心地为你的工作提供支持,希望比以前做的更好。

我已经看过大家2002年六月末所做的工作进展,祝愿大家的研究都进展顺利,这样一篇好的论文就胜利在望了(至少达到《神经科学杂志》的水平),如果那时你们还不能达到这个目标,那我就不得不请你在8月底离开我们实验室。

作为一个科学家,你必须把一切都献身给科研事业。

\subsection{沈向洋: You are how you read }
沈向洋博士定于5月14日北京时间早上8点进行的“You are how you read”演讲直播是在全球创新学院(GIX)课程的分享,



5月14日,沈向洋博士在全球创新学院(GIX)课程上曾做了一场线上公开课《You are how you read》,分享他对于科研论文阅读、撰写的宝贵经验,引起一时轰动。由于围观网友太多,导致会议系统崩溃,众多网友无法接入观看,不免唏嘘不已。近日,为满足广大网友的需求,微软亚洲研究院将沈向洋博士的报告视频公开。 
这场报告亮点纷呈,引人深思。
“或许你永远不知道你以前读过的书能在什么时候能够派上用场,但请保持阅读,因为阅读的过程也是在你大脑中建立认知的过程。”

沈向洋博士在报告如是强调持久阅读的重要性。
“深度阅读论文,要敢于对论文质疑,质疑论文作者的研究方法、思路、技巧。还要设身处地去想:如果我来写这篇论文,我能用什么方法。”

沈向洋博士如是强调论文阅读中“怀疑一切”的精神。
科研离不开日常读论文;而读论文绝不仅是“下载、打开、阅读”这样的被动操作。我们不妨来看一看沈向洋博士的论文阅读经验。

沈向洋:

在研究生院读书是人生中最美好的时光,在这段时间你可能没有足够的钱,但是却有足够的时间学习。

在读书的这段时间应该学习一些重要的技能,比如有效的阅读、写作和演讲。阅读是每一个人生活中必不可少的部分,尤其是对正在读书的研究生来说,他日常的任务可能就是读一些论文,看一些期刊等,但是高效的阅读并不容易。

在过去的20年的时间里,我培养了三十多个博士生,有一些学生能够出色地完成阅读工作,但也有些学生觉得有困难。阅读论文确实很难,这里面可能有很多的原因。首先可能是论文本身写的不是特别好,因为大多数作者的母语并不是英语,而英语又是大多数研究论文的官方语言,所以一些作者在论文语言把控上欠缺火候。

我也是把英文作为第二语言的人,我的论文写得也不够好,现在回过头看看我早期写的论文,真希望当初自己不要写那些论文。

第二个原因是在读论文的时候,读者需要对论文主题有很深的背景知识。

还有,当你在阅读中遇到困难的时候,应该从哪里寻求帮助呢?可以向谁寻求帮助?能够获得什么样的帮助?

另外,现在在网上很容易找到与你阅读相关的东西,所以坐下来长时间专注于阅读一篇长文章或者一本书变得比以前困难很多。

JonathanShewchuck教授在CMU攻读博士学位时候曾经写下过一句话:从所阅读的论文中提取中心思想,就像一句谚语所说“从针眼里吸出一头骆驼”(sucking a camel through the eye of the proverbial needle)。这个比喻非常恰当,以至于我读到这句话的时候也是眼前一亮。

相关文章链接:http://www.cs.cmu.edu/~jrs/sins.html

在那篇文章中,Jonathan指出了CS和数学领域写作的三个常见错误,分别是“祖母式”的引言(意思是引言絮絮叨叨,没有直入主题)、段落式的目录结构、不切题的结论。当然,这三个观点对于写作非常重要,但我们今天的主题不是它,我们今天的主题是:如何阅读。

说实话,有些论文写得确实不好,但写得不好的论文你也要读,因为有时候你没有选择。其实,论文阅读之所以难,最主要的原因是阅读和写作脱节,也就是作者和读者之间的脱节。作者一心想把东西“拿”出来,而读者一心想从文章中得到一些东西,这里面必然有理解的偏差。几千年来,自从语言出现并开始写作至互联网(以及社交网络)出现之前,一直有此困境,即写作只是从作者到读者的单行道,没有循环反馈渠道。例如,直到现在人们还在争论孔子写过的那些文字到底是什么意思。

仔细想想,这类似于香侬的信息论原理,论文是作者和读者之间交流信息的渠道,实际上主要是单向传输。更为具体一些,写作就像编码,阅读就像解码,所以需要一本Codebook来介绍“编码”到“解码”所需要的知识,也即作者按照这本Codebook中的规范进行“编码”,阅读者则用它来“解码”。
其实,香侬理论只是概括了阅读与写作的一部分,真实的阅读往往超越了传统的“传输-压缩”框架,它更多的是一个反复的理解过程。在这个过程中,读者将作者的意图(信息)解释成能够理解的片段,随后这些片段被构建到读者脑中的认知模型里。所以,阅读等同于理解,不同层次的阅读对应不同层次的理解——深度阅读导致深度理解,浅显的阅读对应的必然是浅显的理解。

知道了阅读的本质,那么我们应该如何对待阅读呢?不同需求应该对应不同的阅读层次,你可能需要快速浏览,可能需要仔细阅读,如果你的导师需要你复现论文中的细节,那么你就需要非常深入地研读。

{\bf 快速阅读:划分结构层次}

对于快速阅读,一个小的技巧是图文浏览。因为一些好的论文必然是图文并茂,所以只要弄清楚论文中表格和图片的标题和注释,就能够获得这篇论文八、九成的信息。
对于仔细阅读,你的心态必须是批判和创造的。精读一篇论文,首先要对论文进行否定、质疑,仔细挑毛病;其次,对论文有了足够的了解之后,如果发现论文中提到的想法非常优秀,那么要创造性地思考你能用这篇论文做什么。

除了阅读的方式,读者还要理解所读的论文是如何写出来的。因为一篇好的论文在逻辑上是层层递进的,不仅能够传达信息也能够激励读者。所以作者在写的时候也是有结构化的逻辑性思考程度的。总体上作者会思考:这篇论文的真实任务是什么、研究发现是什么;论文的贡献是什么等等。
相应的,读者在读论文的时候也应该要有逻辑,首先要清楚论文中的表达是否是我想要学习到的;其次,我能从论文中学到多少呢;最后,这篇论文的背景是什么——是什么样的背景让这篇论文变得重要和有趣。

论文结构化的层次只是微观层面,在宏观层面上,读者还需要了解论文的类型:1、提出问题型论文;2、解决问题型论文;3、阐述和调查型论文。

讲完了论文的类型和逻辑结构,接下来进入关于快速阅读和仔细阅读的细节介绍。著名科学家Don Geman曾提到:一篇论文实际上有标题、摘要、引言、论文主体(The restof the paper)组成,论文四个部分的每一部分都需要花同样的时间进行撰写。所以,对论文进行快速阅读需要着重关注论文的标题、摘要和引言,如此便能了解论文是否值得读、能够从论文中读到些什么。
论文最重要,也是最难写的部分是引言。我的前同事,微软美国研究院的著名图形学专家Jim Kajiya在他的一篇著作”How to get your Siggraphpaper rejected”中强调:你必须要下功夫把引言写好,写到审稿人容易看懂。这篇论文是关于什么的,它解决了什么问题,迷人之处在哪,有什么新的东西,为什么那么神奇。

《How to write a good paper》:
\url{https://www.cc.gatech.edu/~parikh/citizenofcvpr/static/slides/malik_write_good_paper.pdf}

如何读论文的引言?举个例子,我最近在读《拥有伦理学:企业逻辑、硅谷与伦理学的制度化》(Owning Ethics:Corporate Logics, SiliconValley, and the Institutionalization of Ethics),在引言部分,我问了Kajiya的几个问题。

首先这篇论文讲的是高科技公司在道德方面做了什么;其次,解决的问题是现在的实践(拥有道德)做法;再者,迷人之处在于最近的丑闻和技术后冲(techlash);这篇文章的最新之处是对17位“道德拥有者”进行了采访;最后,整体的逻辑是道德所有者在令人担忧的动态中运营公司。所以,一旦把引言分解成这些东西,然后你就可以决定:这是有趣的吗?应该继续读下去么?


如何读摘要?计算机学科论文中的摘要一般是非常枯燥的,如果一行一行的读,那么将会更加枯燥。所以要读摘要的时候,将其分解并加上一些有趣的亮点,便可能利于阅读。

举个例子,在《可信度指标对社交媒体新闻分享意愿的影响》(Effects of Credibilit Indicators on Social Media News Sharing Intent)论文中摘要部分,主要讲了社交媒体有可能传播假新闻,对这些媒体加上可信度标记可以减少人们分享假新闻倾向。了解文章的主题之后,然后你需要问自己这篇论文真正的发现是什么?通过继续阅读,你发现论文探究了四种可信度指标,每种可信度指标都能够减少人们分享的倾向,一些审查工作也能带来积极的效应,进一步发现,人口统计和个人特征以及社会特征会对可信度指标产生显著差异的影响。

了解了这篇论文研究成果之后,接着问自己:我为什么要关注这个问题。接着读发现,原来这篇论文的研究成果对遏制虚假信息的传播有着重要的影响。考虑到目前美国信息传播的现状,这对节省研究员的学术精力非常重要,所以对于我来说,我应该关注这个问题。

接下来简单介绍标题如何读,标题一般只有一句话。从整篇论文的排版的角度来看,在一页半的引言、四分之一页的摘要以及八页的双栏正文面前,只有一句话的标题显得不太“重要”。但是标题是总结、抽象的概括。除了论文标题,图表的标题也是非常抽象。我曾经尝试只用两个高度概括的词,并以ing结尾来做Siggraph论文的标题,例如Plenoptic sampling、Lazy Snapping、Poisson matting。所以对于读者来说在读这些抽象的词或句子的时候,只有花费较多的时间才能读懂论文作者的意图。

以上是快速阅读的一些技巧,下面转向仔细阅读。



{\bf 仔细阅读:批判思维}

以评判性阅读开始,带着质疑的心态问问题。如果作者论文中声称解决了一个问题,那么你就要在心里问自己:论文是否正确、真正地解决了问题?作者论文中所用方法是否有局限性?如果所读的论文没有解决问题,那么我能解决么?我能采用比论文中更简单的方法解决么?所以,一旦进入仔细阅读的状态,要在读论文之前对自己说:这篇论文可能有问题,我要找出来。


批判性阅读可能非常难,也可能占用你很多的时间,你可能在读的过程中被卡住,但不要惊慌失措,要坚持下去!多找一些背景知识阅读,多做笔记,多在网上搜索相关论文,然后再次通读所卡壳的论文,并试着把它与其他论文相联系起来,如此坚持下去,定能渡过难关。

另一个建议是,找熟悉这方面工作的人帮忙,让他们解释你遇到的难点。但你要意识到找人帮忙也可能遇到困难,一方面,你要找谁帮忙,导师?师姐?学长?另一方面,如何能让他们愉快的接受你几分钟、十几分钟、甚至接近一个小时的咨询?这一点,对于我来说非常幸运,我在CMU读研究生的时候,我身边的同学和教授都非常友好,尤其是一个比我稍微年长的学长,他的学识比我丰富许多,每当我问他一些论文方面的事情时候,他总能给我令人惊叹的答案。我从他那学到的其实不光是我应该看哪些论文,更多的是不应该看哪些论文。他会告诉我某人的论文你不要读,因为那会扰乱你的思路。
 
 
 {\bf 创造性阅读:积极思考}
 
 好了,现在你已经知道了如何进行批判性的阅读,以及确信哪些论文值得读下去,甚至在读的过程中可能会产生一些好的想法。那么接下来要做一些改进,从而进入创造性阅读层面。
 
这时候你要问自己:在我所读的论文中,作者有哪些点还没有想到?如果我现在做这项研究,我能做的新事情是什么?创造性的阅读需要把你所读的论文和其他相关的论文建立联系,从而产生一些新的想法,这些想法可以支撑你进行三个月到五个月的研究。如果你真正想理解你所读的论文,那么就写一个摘要吧,最好做一个口头展示,这样你会发现,只有把东西写下来或者说出来才能真正深刻理解。如果你能做一个报告,那就更好了,因为做报告的时候,别人可以问你问题,这会强迫你理解所读的论文。

在做这个演讲之前,我曾经向我的同事、学生询问了关于论文阅读有哪些问题可以“问自己”,上面这张图片是一个总结,图片的上半部分是比较客观的问题,包括论文的核心观点是什么?主要的局限性是什么?代码和数据是不是可得的?论文的贡献是否有意义?论文中的实验是否足够好?
图片的下半部分是比较主观的问题,包括我错过了什么相关论文么?这对我的工作有何帮助么?这是一篇值得关注的论文么?这个研究领域的领头人是谁呢?其他的人对这篇论文有何看法呢?如果有机会见到作者,我应该问作者什么问题?

当你在阅读论文的时候如果能回答出上面列出的问题,我相信你会对你所读论文有非常深刻的理解。

此外,还有一些工具能够帮助阅读,例如谷歌以及必应网页搜索引擎、谷歌学术、arXiv等能够搜索到你想要的论文;OneNote可以帮助你做笔记;CliffsNotes作为美国知名的学习指南网站,能够提供文献学习指南;Mendeley是非常优秀的参考文献管理。另外也强烈建议大家使用在线论坛讨论论文,增加学术交流,增强对所读论文的理解能力,但是遗憾的是,尚未有非常适合讨论论文的在线论坛,现在的一些社交网络产品或许很棒,但是它并不是为了学术研究目的而设计,毕竟学术论文的“非有趣性”不适合社交产品的调性。

最后,上面有几个小贴士希望你注意。我最想强调的是要养成写小总结的习惯,最好能做报告,这样真的能够增加你对所读论文的理解。

或许你永远不知道你以前读过的书能在什么时候能够派上用场,但是请保持阅读,因为阅读的过程也是在你大脑中建立认知的过程。


\subsection{陆品燕: 用一年时间,建设国际一流理论计算机研究中心}
%  https://www.leiphone.com/news/201710/O96kJcQQtfXtpExH.html?uniqueCode=SNEvrPXbuaOFHxbX

上海财经大学理论计算机科学与研究中心官网:http://itcs.shufe.edu.cn/

陆品燕,上海财经大学理论计算机科学研究中心(ITCS)主任,CCF 青年科学家奖得主。他的主要研究方向是理论计算机,并注重与其它学科的交叉,包括自然科学中的统计物理以及社会科学中的经济学与社会选择理论等。有 50 余篇科研论文在 STOC、FOCS、 SODA、 EC 等顶级计算机理论会议发表,荣获ICALP 2007、FAW 2010、ISAAC 2010 等重要国际会议最佳论文奖,多次担任 STOC,FOCS,ICALP等顶级国际会议的程序委员会委员。

AI科技评论:在刚刚公布的国际算法顶级会议 SODA 2018 的接收列表中,ITCS 有三篇论文被录用,占据中国大陆论文的一半,这对于成立刚满一周年的 ITCS 是非常了不起的成就,这一年 ITCS 取得了哪些成果?

陆品燕教授:ITCS 在一年多的时间里从零开始到初具规模,取得的成果和成长的速度都得到了圈内外同行的普遍认可,甚至超过了我自己预期的速度。除了你提到的一些具体的论文等学术成果,我觉得这一年主要的成果是团队建设。

现在 ITCS 已经有 5 位全职的研究人员,研究领域既涵盖理论计算机的核心领域包括算法与复杂性等,也包括很多理论计算机与其他领域的交叉,比如经济学,人工智能,复杂网络,统计物理等等。这些年轻的研究员们已共计发表理论计算机的顶级会议与期刊 STOC 11 篇,FOCS 6 篇,SODA 22 篇,EC 9 篇,SIAM Journal of Computing 8 篇等,另外还在很多其他领域的重要会议及期刊发表论文,比如 NIPS,WWW,IJCAI,AAAI,Physical Review E,Scientific Reports,Mathematics of Operations Research,Games and Economic Behavior 等等。

此外,ITCS 还拥有一个由 10 余名海内外知名学者组成的讲席教授组,每人每年至少会在 ITCS 访问一个月,这个讲席教授组囊括了全球华人青年理论计算机科学家的半壁江山。

一年多来,我们共接待来自全球各地的研究人员 50 余人次,组织学术讲座 50 余场,举办 4 次研讨会,4 次暑期课程。在这里,非常感谢我的同行朋友们,是你们各种形式的支持,使得这一切在非常短的时间内成为可能,使得 ITCS 虽为后辈,却已成为了计算机科学理论界不可忽视的一个幸运儿。

AI科技评论:作为研究中心主任,您在 ITCS 的成立仪式上提出了「三个平衡」的中心发展理念(理论与应用,中国与世界,理想与务实)。在过去一年中,您认为 ITCS 是如何在这三者间取得平衡的?

陆品燕教授:有些读者可能没有读过我关于三个平衡的讲话稿,这里引用一小段以方便读者理解它的大概含义:

(理论与应用)如果太偏理论了,完全无视应用,很可能走得越来越偏、越来越窄,最后变得没有活力、没有源泉;如果太偏应用的话,则可能完全丢掉理论及数学这个灵魂,完全没有自己的根基与特色,不断追逐应用中一轮又一轮的所谓热点,最后不能在历史的潮流中留下真正有意义的东西。

(中国与世界)如果太偏中国特色华人学者,那很可能就成为一个典型的中国特色的研究中心,与国际一流严重脱节,最终导致吸引过来的海外华人也是一帮三流的学者,真正一流的活跃在科研第一线的华人学者也不愿意过来这边浪费时间;而另一方面,如果纯粹去追求已经成名成家的老外学者,最后效果可能只是撑了门面,没有真正起到效果,而且整个中心会变得非常脆弱,不可持续发展。

(理想与务实)太纯粹的理想主义而不解决实际问题和困难,很难吸引过来真正优秀的学者,因为真正一流的人才都有很多的选择,在个人的层面,极度的不切实际的理想主义往往让一个人变得喜欢抱怨、愤青;但如果太务实,很可能把研究中心办得越来越大、越来越商业化,不断追逐大项目、大奖项,忙碌于申项目、填表格、拉关系而完全忘记这一切的初衷是为了更好的科研更好的教育。」

正像我当时就表达的,这三个平衡是我根据很多经验和教训总结出来的,而不是心血来潮随便说说的。在过去一年的实践中,我觉得我们基本上没有偏离这三个平衡太远,而且我们也会一直坚持下去。

但这一路走来真的非常不容易,阻碍和诱惑并存,需要的是一种坚持和定力。这个过程中,我最要感谢的是上海财经大学及信息学院的各位领导和老师给了我们一个非常宽松而自由的环境,因为除了自身的定力和内在修养,外界的环境对于维持这个平衡也是非常重要的。这个外界环境主要有两个:各级政府及学校的教育管理部门所给出的制度与绩效评价,以及包括你们在内的各个媒体所营造的舆论与社会评价。

从教育管理部门的角度,要认识的不同的学科间有具体的差别,避免一刀切,给予每个高校,每个学院,每个科研工作者足够的自主权和自由度,在评价的时候主要依靠国际同行评议而不仅仅是客观数据。

这里举两个具体的例子来说明学科的差异。比如,现在国内很看重 SCI 杂志论文,这对很多学科是有参考意义的,但计算机方向其实更看重会议论文,所以当计算机学科与其他学科横向对比的时候,如果单纯从 SCI 论文这样的客观数据来比就很成问题;再如,在计算机学科内部的比较时,大家都看重顶级会议的论文,这一点没有问题,但是大多数计算机应用的学科都很注重第一作者论文,第一作者与非第一作者是个很大的差别,但是理论计算机科学的国际惯例是作者按字母序排列,排序完全不体现贡献大小,所以是否第一作者根本没有任何区别,但大家用第一作者论文这个客观数据来比较理论计算机学工作者的时候就很成问题。

从媒体的角度,适度的关注科研,关注科学家对于大众,对于学术圈都是很有意义的,可以给大众普及科学,也同时营造一种尊重科学尊重科学家的氛围。但科学及其发展是很严肃的事情有其内在的发展规律,现在有些媒体动辄就是「重磅」、「突破」、「诺奖级」,形成了一种浮躁氛围,甚至以娱乐化方式来进行科学与科学家故事的宣传,对于学术界的发展以及大众对于学术界的真实了解都是不利的。不仅对于大众不利,也不利于科研工作者心平气和地科研。

AI科技评论:ITCS 研究的主要核心内容包括算法与复杂性等,也包括很多理论计算机与其他领域的交叉,但这些领域势必会涉及与现实生活的交融。像 ACM EC 这种以往偏理论的会议,今年的不少论文与 workshop 议题也已经开始延伸到实际应用领域,也做出了不少的尝试。您认为这一变化趋势会对理论计算机的创新性研究产生哪些影响?

陆品燕教授:其实从建立之初,理论计算机科学就处在不同领域交叉的前沿。姚期智先生在 2015 年的计算机科学联合大会(FCRC 2015)上作了一个题为「Interdisciplinarity: A View from Theory of Computation」的邀请报告。姚先生在报告中很好的阐述了理论计算机科学的这一特点,并以他自己在量子计算、计算经济学等领域的研究来具体说明,我当时听了很受启发。我们在 ITCS 的研究也非常注重理论计算机科学与其他学科的结合。

在 10 月 15 日,我会在全国理论计算机学术年会上作一个「理论计算机——一门交叉学科」的特邀报告,仔细地阐述理论计算机科学的这个特点。这里所说的学科交叉不是很简单地把一个学科的结果或者方法应用到其他一个学科,而是两个学科最本质的思维方式结合在一起,产生一门全新的学科。计算思维是当前最重要的一种思维方式,它在自然科学,社会科学,工程技术学,数学等方向都有深刻的应用。

我经常觉得自己作为一个理论计算机科学家,特别是一个注重交叉学科的理论计算机科学家非常幸运。我的朋友与合作伙伴中有数学家,物理学家,经济学家和计算机应用科学家与工程师,和他们都有很好的共同语言,虽然不懂细节,但可以比较恰当的理解与欣赏他们的工作,但也可以知道他们的局限。他们之间因为距离太远往往有些太高或者太低的极端偏见。

比如我的工程师朋友们对于数学家的认识往往偏向两极中的一极:要么觉得数学家绝顶聪明无所不能,如果来做计算机科学的研究肯定特别轻松;要么觉得数学家做的东西完全是自娱自乐没有任何用。

数学家对于计算机技术的认识也是类似的两极:要么觉得计算机科学都是一些肤浅的平凡的东西,没有什么深刻的内涵;要么觉得计算机技术特别是人工智能技术如此强大,机器人马上就可以抢了数学家的饭碗。

这种彼此太高或者太低的认识对于相互合作是不利的,理论计算机科学处在这些学科的交叉位置,和两边都有共同语言,这对于交叉学科的研究是非常重要的。反过来,这样一种交叉对于理论计算机科学本身的发展也是非常重要的。

AI科技评论:目前中国在国际理论计算机领域处于怎样的科研水平?在中国建立一个国际一流的理论计算机科学研究中心,您觉得目前还有哪些挑战和机遇?

陆品燕教授:中国理论计算机的发展基础比较薄弱,但近些年,特别是姚期智先生回国之后,有了长足的进展。现在正处在一个机遇与挑战并存的阶段,下面从两方面具体展开讲。

我觉得最大的机遇是我们已经有一个很好的基础,在世界各地从事理论计算机科学的青年学者与学生人数在这十多年中有了很大的提高,已经达到一个临界数量(critical mass)。

记得自己在 2007 年第一次参加 STOC 的时候,基本上看不到什么中国面孔,但现在去参加 STOC/FOCS/SODA 这三大理论会议的时候,中国学者与学生可以凑好几桌,虽然比例还远远没有达到数据挖掘,人工智能等华人占了半边天的会议,但也已经比之前有了很大的进步。

这个人群基础是在中国建立一个国际一流的理论计算机科学研究中心的重要条件,这也是 ITCS 建设与发展中特别倚赖的人群,华人学者中的大多数在过去的一年中已通过各种形式访问过 ITCS,我们中心已经成为圈内同行回国的默认据点,其中一部分更是和我们有长期的伙伴关系,每年至少在我们这边访问交流一个月。当然,另一方面,我相信 ITCS 对于他们也是有帮助的,正像我在中心的揭牌仪式上所说的,ITCS 是华人理论计算机科学家共同的家园。

而其中,最大的挑战是在中国大陆整个科研梯队还没有形成一个稳定的金字塔结构:一批大基数的不错的科研工作者,和逐层递进的优秀科学家,共同构成一个良性的流动的科学共同体,大家有比较一致的价值观和共同语言。

国内在各个方向都已经涌现出一些完全达到国际一流水平的科学家,但这些杰出的个人、研究中心和国内主体的教育科研体制还没有形成一个良性的互动。比如最近引起大家关注的许晨阳、颜宁从海归到归海,各自肯定有各自具体的很多原因我们没必要细究,但和这整个科研大环境的氛围肯定是有关系的。 坊间流传着一种说法,国内一流和国际一流是两个很不一样的概念,有时候甚至国际一流比国内一流还要容易达到一些,这就是一个很大的问题。

AI科技评论:从工作七年的微软亚洲研究院转向上海财经大学就职,并建立研究中心,您曾提及这是您一直以来的梦想。在我们惯常的印象中,上海财经大学可能并不是一个在计算机科学领域非常拔尖的高校,能否分享下当时做出这个决定的心路历程?

陆品燕教授:这是很多人问过我的问题,我一般的回答是「上财的理想主义情怀与务实的态度很完美的契合了我自己的梦想。」我在这里再仔细解释一下。

首先上财一直有一种追求卓越的理想主义情怀,很多人可能不知道,上财是国内最早一批实行常任轨制度的学校之一,已经十几年了,制度上非常成熟,也非常接近美国的 tenure track,而没有做一些不伦不类的改良。特别在人才的引进上,特别注重国际同行的评价,而不仅仅是看一些客观的数据,也不是很看重资历和国内的各种帽子,希望我们产出的成果也首先是国际同行的认可。

从务实的角度说,樊校长和我接洽的第二天就联合科研处,人事处,后勤处开了一个部门协调会议专门讨论能够为我的研究中心提供的支持,效率特别高,而且在之后的落实过程中对于之前承诺的支持完全不打折。这些实实在在的支持是 ITCS 在这一年中得以快速发展的重要保证。

至于你说的上海财大不是一个以计算机科学见长的一个学校,这确实是事实,也给我们的发展造成了一些障碍。但我相信这是可以克服的,学校也在大力发展整个计算机学科,而且从某种意义上说,我们可以利用这种后发优势,直接以一个国际一流的标准和目标来发展和建设。

AI科技评论:目前有很多学者「下海」工业界,而您反而是从工业界的研究院转向学术界,有种「逆流而上」的感觉,您觉得学术界与工业界各自的吸引力在什么地方?

陆品燕教授:这是一个很好的问题,对于计算机科学这样一门应用导向的学科,工业界和学术界的双向流动是很正常的现象,对产学研的结合很有好处。但另一方面确实也造成了不少问题,特别是对于基础研究。但具体到我的个例来说,其实我从来没有觉得自己离开过学术界,微软亚洲研究院的环境其实非常像一个大学的研究所。

大面上来说,我觉得工业界的优势是能把一件事情做到极致,甚至真的能做成产品,因为公司在人力物力的调配上更集中更迅速。但学术界的优势是能够发挥每一个科研人员自己主观能动性,根据自己的兴趣自由地选择科研题目,更有可能做出一些原创性的工作。

AI科技评论:您在 2015 年接受 CCCF《动态》栏目采访时评价自己是一个「工科出身、有理科情结的哲学思考者」,那么如今两年过去,您对自己的定位是否有改变?这样的特质在科研上的最大优势是什么,又获得了哪些成果?

陆品燕教授:实际上早在读博期间,我就是这样来定位自己的,现在差不多十多年了,也并没有太多变化。我觉得这样一种特质是我个人的特点,很多时候对我的科研选题、科研品味会产生影响。

理科情结让我喜欢做一些在数学上深刻的基本问题,而不仅仅是一些搭积木式的增量式研究;工科出身让我也关注选题的有用性,相关性,而不仅仅是把它们作为单纯的智力游戏;哲学的思考方式,让我更习惯从宏观上,整体上来把握问题和方法论,而不仅仅是在技术的层面。

当然,这一年多来创建 ITCS 的经历对于我自己的特质与视角也有一些补充,我开始学着从一个管理者、团队建设者的角度来思考一些问题,而不仅仅是自己的科研工作。人生是一个体验与自我完善的过程,我相信这样一种新的特质与视角对于我之后的发展是重要的,我也正在努力学习中。



\section{Nature力荐:读博士的4条“黄金法则”}

史蒂文·温伯格(Steven Weinberg,1933年5月3日-),生于纽约,美国物理学家,1979年获诺贝尔物理学奖。


《Four golden lessons》是美国物理学家、诺贝尔奖(1979)获得者Steven Weinberg发表在Nature-scientist 上的一篇文章,文章中,温伯格为即将进入科研领域的研究生总结了四条箴言。



文章英文原文深入浅出,行文优美。是大师温伯格近50年科研生涯的感悟和总结。堪称经典,读后获益匪浅,得到很多科研大牛及导师力荐。



国际著名生物学家,美国国家科学院外籍院士颜宁看到后,在其科学网博客上感叹道:

“竟然是第一次看到这篇11年前的短文,写的真好。”



以下为《Four golden lessons》的中文译文:

golden lessons1:

没人知道所有的事情,你也无需如此


我取得学士学位的时候,距今已经很遥远了。那时,物理学文献对我来说,就是一片广阔而未知的海洋。在开始任何研究之前,我都想仔细研究它每个部分的内容。因为,如果不知道这个领域的都已经做过的每个研究,我又如何能开展研究呢?幸运的是,读研究生的第一年,我运气很好。尽管我满心焦虑,但却得到了资深物理学者们的引导,他们坚持认为,我必须先开始研究,在研究过程中获取相关的知识。这就好比游泳,要么选择淹死,要么奋力游过去。令我惊讶的是,我发现这样做真的有用,我很快便获得了一个博士学位。尽管拿到博士学位时,我对物理学几乎一无所知,但是我确实学到了一个重要道理:没人知道所有的事情,你也无需如此。

golden lessons2:

向混乱进军,因为那里才大有可为


如果继续用游泳来打比方,我学到的另一个重要道理就是:游泳时不想被淹死,就应该到湍急的水域去练习。上世纪60年代末,我在麻省理工学院教书时,一个学生告诉我,他想去研究广义相对论,而不是我本人研究的专业粒子物理学。他的理由是,前者的原理已广为人知,后者却好似一团乱麻。在我看来,他所说的恰好是做出相反选择的绝佳理由。粒子物理学还有许多创造性工作可以做,它在上世纪60年代确实像一团乱麻,但从那时起,许多理论和实验物理学家逐渐厘清这团乱麻,把一切(几乎一切)纳入一个我们现在所说的一个叫做“标准模型”的理论。所以我的建议是:向混乱进军,因为那里才大有可为。

golden lessons3:

原谅自己浪费时间


我的第三条建议或许最难被接受:那就是原谅自己浪费时间。学生们只被要求回答教授们(当然,不包括残忍的教授)认为存在答案的问题。但是,这些问题是否具有重要的科学意义也无关紧要——因为解答这些问题的意义只为了让学生通过考试。但在现实世界中,你很难知道这些问题是否重要,而且在历史的某一时刻你甚至无法知道这个问题是否有解。二十世纪初,包括洛伦兹(Lorentz)和亚伯拉罕(Abraham)在内的几位重要物理学家试图建立一个电子理论,部分原因是为了解释为何地球在以太中运动所产生的效应为何无法被探测到。我们现在知道了,他们在试图解决一个错误的问题。当时,没人能提出一个成功的电子理论,是因为那时还没发现量子力学。直到1905年,天才的科学家阿尔伯特·爱因斯坦才发现,需要研究的问题应该是运动对时空测量的效应。从这一思路出发,他才创建了狭义相对论。你永远也无法确定研究什么样的问题是正确的,所以你花在实验室或书桌前的大部分时间都会被浪费掉。如果你想变得富于创造性,那你就应该习惯自己的大部分时间都没有创造性,同样应该习惯在迷路在科学知识的海洋里。

golden lessons4:

学习科学发展史,至少你研究的领域要了解


最后的建议是:学习科学发展史,至少,你研究领域的历史要了解。最起码,历史可能为你自己的科研工作提供一定帮助。比如,过去和现在的科学家们常常会因为相信像培根(Francis Bacon)、库恩(Thomas Kuhn)、波普尔(Karl Popper)等古代哲学家们所提出的过分简化的科学模型而被阻碍。而挣脱古代哲学家思想束缚的最好方式,就是了解科学发展史。
更重要的是,对科学史的了解可以让你更加清楚自己工作的价值。作为一名科学工作者,你可能永远也不会变得富有;你的亲戚和朋友或许也永远不会懂你在做什么;更进一步,如果你在像高能粒子物理学这样的领域工作,你甚至无法获得做那种立竿见影的工作所带来的满足感。但是,如果你意识到你的工作是世界科学历史的一部分,你就能获得极大的满足感。

回望百年前的1903年,谁是英国首相,谁是美国总统都已经不重要了。我们看来真正具有重要意义的,是卢瑟福(Ernest Rutherford)和索迪(Frederick Soddy)在麦吉尔大学揭示出了放射性的本质。这项工作当然有实际应用,但更重要的却是其中的内涵。对放射性的了解使得物理学家终于能够解释,为何历经数百万年后,太阳和地球的内核仍然炽热。从前许多地质学家和古生物学家认为太阳和地球有着极为巨大的年龄,这就消除了科学上对此最后的异议。自此以后,基督徒和犹太教徒要么不得不放弃相信《圣经》所记载的教义,要么不得不承认自己与理性毫不相干。从伽利略到牛顿,再到达尔文,再到现在的科学家,他们的研究一次又一次地削弱了教条主义的禁锢,而卢瑟福和索迪的工作只是其中的一步。当今,只要随便阅读一份报纸,你就会知道这项任务还未完成。不过,这是一项令社会文明化的工作,科学家应该为此工作感到骄傲。

Article Source: Nature 426, 389 (27 November 2003)

doi:10.1038/426389a

Scientist: Four golden lessons by Steven Weinberg

http://www.nature.com/nature/journal/v426/n6965/full/426389a.html

来源 | Nature,列文虎克网


\subsection{ 包云岗:无题}

\url{https://m.weibo.cn/status/4524524207258266?#&gid=1&pid=2}


\section{沈向洋、华刚:读科研论文的三个层次、四个阶段与十个问题   }

7 月 18 日(上周六),微软亚洲研究院“沈老师带你肝论文”暑期科研训练班在线上进行了开班仪式。



“沈老师带你肝论文”是微软亚洲研究院为实习生们特别打造的专属科研训练。学会阅读论文是科研工作的第一步,在未来的20天内,美国国家工程院外籍院士、清华大学高等研究院双聘教授、微软公司前执行副总裁沈向洋博士将领衔顶级导师团,带领30名幸运鹅学习如何正确、高效地进行论文阅读和批判性思考,在肝论文过程中练就坚实的科研基本功。

开班仪式上,沈向洋博士与同学们分享了他阅读、撰写科研论文的宝贵经验。阅读论文有消极阅读、积极阅读、批判性阅读和创造性阅读这四个阶段,也有速读、精读与研读这三个层次,有价值的论文阅读将帮助研究者建立认知模型,找到有价值的研究想法。

而后,Wormpex AI Research 副总裁兼首席科学家华刚博士分享了他对于初级科研工作者如何通过论文阅读获得快速成长的思考,同时详细介绍了读论文应回答的十个问题,本次科研训练班也正将带领同学们学会思考和解答这些问题。


下面,我们将与大家分享沈向洋博士和华刚博士的演讲,希望为做科研的你带来思考和启发。


沈向洋

美国国家工程院外籍院士

英国皇家工程院外籍院士

微软公司前执行副总裁



沈向洋博士主要专注于计算机、视觉、图形学、人机交互、统计学习、模式识别和机器人等方向的研究工作。他所设计的四分树样条函数算法是世界上最好的运动参数估计算法之一。他已发表关于计算机视觉、计算机图形学、图形识别、统计学习和机器人科学方面的数百篇论文,拥有超过 50 项美国专利。


华刚

Wormpex AI Research

副总裁兼首席科学家



华刚博士是 IEEE Fellow,IAPR Fellow 和 ACM 杰出科学家。他的研究领域包括计算机视觉、模式识别、机器学习和机器人技术等。在加入 Wormpex 之前,华刚曾担任微软计算机视觉科学主任以及史蒂文斯理工学院副教授。


沈向洋博士:
如何以正确方式
打开一篇科研论文?


很高兴有这样一个机会跟各位同学、VC 组的研究员们交流,也非常感谢华刚跟我一起准备这个演讲。



我想今天听报告的大多数学生应该是在研究生院阶段,我非常喜欢这个阶段的生活,因为这可能是你一生中时间最充足的阶段,以后工作了就会非常忙。我想强调的是在这个阶段,你应该多读书、多读文章。如果你决定要从事科研工作,就需要不断地学习、理解和消化知识,再过渡到自己创造知识、散布知识。



多年前,我到微软亚洲研究院后就成立了 Visual Computing 组,对它有着深厚的感情。这几年我也一直在思考和解决一个非常重要的问题——阅读和理解之间的不匹配。通过科研论文的角度去思考阅读和理解之间的关系,是整个人类智能中非常重要的部分。下面,我想分享自己对做科研方面的一些体会,特别是怎样读 paper 和写 paper、怎么样更好地做科研。

我认为好的研究员有几个特质,首先要 open-minded——这个世界只有想不出来的东西,没有做不出来的东西,要有批判性思考的能力;其次是要努力工作;还有要不断更新自己的知识面,要读很多最新的东西,然后思考、交流,这样才能慢慢把自己学到的东西用起来。

今天主要想跟大家分享我在读科研论文方面的一些心得。读文章有几个阶段,最简单的是所谓的“消极阅读”(passive reading),即大概知道文章讲了什么;然后是“积极阅读”( active reading),主动思考这些知识有什么用;然后是“批判性阅读”(critical reading),思考这篇文章是否言之成理理;最后是“创造性阅读”(creative reading),搞清楚文章对接下来的工作有什么帮助。



我认为在读研究生期间需要掌握三个非常重要的技能:阅读、写作和展示。实际上这三件事情有内在的逻辑关联。我有个同事 Simon Peyton Jones 对此提了一些建议,大家可以到微软研究院网站观看他的视频“how to write a great research paper”和“how to give a great research talk”。


读论文为什么这么难?



如果大家决定做科研,那读论文就是必修课。为什么读论文这么难呢?


首先,大多数科研论文本身写的不是特别好,大多数作者的母语并不是英语,而英语又是学术界的官方语言,所以一些作者在论文语言把控上欠缺火候。我回过头来看自己早年写的几篇 paper,有时候会希望自己没有写过它们。文章写得不好只是一个客观原因,论文难读的第二个原因,是读论文时读者需要对论文主题有很深的背景知识储备。第三个原因是在阅读中遇到困难的时候,我们不知道应该从哪里、向谁寻求帮助。第四个原因是读完论文以后,如果我想继续深挖这个主题或者探索研究方向,除了去问导师以外,还可以向谁寻求意见呢?第五个原因,是当今世界有太多的诱惑和干扰,不像我们以前“两耳不闻窗外事、一心只读圣贤书”,在这个有互联网的世界里,长时间专注是一件很困难的事情。



我想跟大家分享这几年我一直在想的一个问题,就是所谓的“disconnect between reading and writing”。人类社会发展到现在离不开获取知识和利用知识,但是目前为止,阅读和写作这两件事仍是脱节的,作者和读者的非直接沟通中必然有理解的偏差。



实际上,这种关系可以用香侬的信息论原理来解释——论文是作者和读者之间交流信息的渠道,主要是单向传输,信息源是作者,而目的地就是读者。写作就像编码,阅读就像解码,所以我们需要一本 Codebook 来介绍“编码”到“解码”所需要的知识,也即作者按照这本 Codebook 中的规范进行“编码”,阅读者则用它来“解码”。

其实,香侬理论只是概括了阅读与写作的一部分,真实的阅读往往超越了传统的“传输-压缩”框架,它更多的是一个反复的理解过程。在这个过程中,读者不断地揣测作者的意图,并将之解构成能够理解的片段,随后这些片段被构建到读者脑中的认知模型里。所以,阅读等同于理解,不同层次的阅读对应不同层次的理解——深度阅读导致深度理解,浅显的阅读对应的必然是浅显的理解。不同需求应该对应不同的阅读层次,你可能需要快速浏览,可能需要仔细阅读,如果你的导师需要你复现论文中的细节,那么你就需要非常深入地研读。



阅读文章的三个层次:速读、精读与研读



了解了阅读的本质,那么我们应当如何阅读 paper 呢?



首先是要有“速读”的能力,快速知道一篇文章讲了什么。其次是要“精读”。精读有两个方面:批判性阅读和创造性阅读。首先要对论文进行否定、质疑,仔细挑毛病;其次,对论文有了足够的了解之后,如果发现论文中提到的想法非常优秀,那么要创造性地思考你能用这篇论文做什么。第三个步骤,我称之为“研读”,比如说自己尝试将文章中的算法实现一遍。



除了阅读的方式,读者还要理解所读的论文是怎样写出来的。一篇好的论文在逻辑上是层层递进的,不仅能够传达信息也能够激励读者。所以作者在写的时候也是有结构化的逻辑性思考程度的。总体上作者会思考:这篇论文的真实任务是什么、研究发现是什么;论文的贡献是什么等等。



相应地,读者在读论文的时候也应该要有逻辑,首先要清楚论文中的表达是否是我想要学习到的;其次,我能从论文中学到多少,能不能找到新的方向与新的课题初稿;最后,这篇论文的背景是什么——是什么样的背景让这篇论文变得重要和有趣。



无论是计算机视觉领域的文章还是泛计算机类的文章,一般来讲,都可以归为以下几类:提出问题型论文、解决问题型论文、阐述和调查型论文、总结型论文。



快速阅读:如何读标题、摘要和引言



接下来我向大家介绍一些读论文的经验。首先是快速阅读。计算机视觉领域的著名学者 Don Geman 曾经说,一篇文章可以分为标题、摘要、引言、论文主体四个部分,每一部分都需要花同样的时间进行撰写。这个说法虽然夸张但是不无道理,因为大多数读者实际上最关注的就是文章开始的两页纸。对读者而言,看完前两页就知道这篇文章是不是值得去读;对 reviewer 而言,看完开头就知道能不能拒绝这篇文章。



我的前同事 Jim Kajiya 是一个非常了不起的图形学专家,他最牛的地方就是基本上从来不和别人合写文章,都是自己独立完成。Jim曾经写过一篇文章“How to get your SIGGRAPH paper rejected”,文中最核心的观点是文章一定要写得易读——这篇论文是关于什么的?它解决了什么问题?迷人之处在哪?有什么新的东西(我一直强调做科研的终极问题就是 what's new,写文章的时候一定要强调文章中有什么新的东西)?巧妙之处何在?


如何读论文的引言?举个例子,我最近在读《拥有伦理学:企业逻辑、硅谷与伦理学的制度化》(Owning Ethics:Corporate Logics, Silicon Valley, and the Institutionalization of Ethics),在引言部分,我问了 Kajiya 提出的那几个问题。首先这篇论文讲的是高科技公司在道德方面做了什么;其次,解决的问题是现在的实践(拥有道德)做法;文章的迷人之处在于最近的丑闻和技术后冲(techlash);这篇文章的最新之处是对 17 位科技公司的“道德拥有者”进行了采访;最后,论文的神奇之处是得出了结论:道德所有者在令人担忧的动态中运营公司。

接下来,我将介绍如何读摘要。计算机学科论文中的摘要一般有固定格式,读起来非常枯燥。所以,要读摘要的时候,将其分解并加上一些有趣的亮点,可能有利于阅读。中国学生的英语一般都不太好,虽然到了研究生阶段,思维方式可能还停留在中文思维到英文思维转换的阶段。对此,一个很好的建议是尝试把摘要翻译成中文,在这个过程当中,你会发现自己有一些细节没有领会到位,第一遍读的时候不见得读懂了。



回到标题,如何读论文的标题?标题一般只有一句话。从整篇论文的排版的角度来看,在一页半的引言、四分之一页的摘要以及八页的双栏正文面前,只有一句话的标题显得不太“重要”。但是标题是总结、抽象的概括。我曾有一个重要发现:高质量的文章通常标题用两个词就能概括,并以 ing 结尾,例如 Plenoptic sampling、Lazy Snapping、Poisson matting。所以对于读者来说在读这些抽象的词或句子的时候,只有花费较多的时间才能读懂论文作者的意图。


仔细阅读:从批判性阅读到创造性阅读



下面我将介绍仔细阅读的一些技巧。


以批判性阅读开始,带着质疑的心态问问题。如果作者论文中声称解决了一个问题,那么你就要在心里问自己:论文是否正确、真正地解决了问题?作者论文中所用方法是否有局限性?如果所读的论文没有解决问题,那么我能解决么?我能采用比论文中更简单的方法解决么?所以,一旦进入仔细阅读的状态,要在读论文之前对自己说:这篇论文可能有问题,我要找出来。这就是批判性阅读。

批判性阅读可能非常难,也可能占用你很多的时间,早期知识储备不够时,读论文的过程中很容易卡壳。常见的建议是找熟悉这方面工作的人帮忙,让他们解释你遇到的难点。但你要意识到找人帮忙也可能遇到困难,一方面,你要找谁帮忙?另一方面,如何能让他们愉快的接受你几分钟、十几分钟、甚至接近一个小时的咨询?他们不一定有那么多时间。所以我一直鼓励大家在研究生阶段一定要跟身边导师、师兄师姐、厉害的同学们搞好关系。等到你自己成为师兄师姐后,也要积极回应学弟学妹们的提问。

除此之外,也建议你们多找一些背景知识阅读,多做笔记,多在网上搜索相关论文,然后再次通读所卡壳的论文,并试着把它与其他论文相联系起来,如此坚持下去,定能渡过难关。



掌握了批判性阅读的技巧后,如何达到创造性阅读的层次呢?这时候你要问自己:在我所读的论文中,有哪些好的 idea?(一般文章中只有一个 idea,好的文章中可能有两个 idea,最了不起的文章可以有2.5个 idea。)搞清楚作者的 idea 以后你要思考,作者有哪些点还没有想到?可以怎么改进?如果我现在做这项研究,我能做的新事情是什么?



如果说批判性阅读是“negative thinking”,那么创造性阅读就是“positive thinking”。创造性的阅读需要把你所读的论文和其他相关的论文建立联系,从而产生一些新的想法,这些想法可以支撑你进行三个月到五个月的研究。如果读到了非常好的文章,不妨写一篇半页到一页左右的 review。最好做一个口头展示,这样你会发现,只有把东西写下来或者说出来才能真正深刻理解。

我一直觉得理解了一个东西以后,最重要的是能够自问自答,这张图片是一个总结,图片的上半部分是比较客观的问题,包括论文的核心观点是什么?主要的局限性是什么?代码和数据是不是可得的?论文的贡献是否有意义?论文中的实验是否足够好?图片的下半部分是比较主观的问题,包括我错过了什么相关论文么?这对我的工作有何帮助么?这是一篇值得关注的论文么?这个研究领域的领头人是谁呢?哪些公司、研究院、实验室值得关注?其他的人对这篇论文有何看法呢?如果有机会见到作者,我应该问作者什么问题?当你在阅读论文的时候如果能回答出上面列出的问题,我相信你会对你所读论文有非常深刻的理解。




有哪些工具可以帮助我们?

我还想跟大家分享一些能够帮助阅读论文的工具,例如谷歌以及必应网页搜索引擎、谷歌学术、arXiv 等能够搜索到你想要的论文,但这些工具都不能真正帮助我们读懂 paper。在阅读的过程中,OneNote 可以帮助你做笔记;CliffsNotes 作为美国知名的学习指南网站,能够提供文献学习指南;Mendeley 是非常优秀的参考文献管理。



另外也强烈建议大家使用在线论坛讨论论文,增加学术交流,增强对所读论文的理解能力,但是遗憾的是,尚未有非常适合讨论论文的在线论坛,现在的一些社交网络产品或许很棒,但是它并不是为了学术研究目的而设计,毕竟学术论文的“非有趣性”不适合社交产品的调性。



最后,我想向大家介绍几个小 tips。我最想强调的是要养成写小总结的习惯,最好能做报告,这样真的能够增加你对所读论文的理解。我想再次强调,大家一定要有耐心,因为阅读就是在你大脑中建立认知模型的过程,虽然不知道今天读的文章未来什么时候能够派上用场,但是请大家保持阅读、建立认知的习惯。




华刚博士:
带着十个问题去阅读和思考



接下来我将和大家分享作为研究员如何在学术领域获得成长的一些经验。这次暑期训练班的初衷是希望帮助到大家建立科学研究工作的认知模型,下面我会引入一种叫做“模板阅读”的方法论。



如同前面沈老师所说的,读 paper 可以分为四个阶段:消极阅读、积极阅读、批判性阅读和创造性阅读。



我认为大家可以通过严格的科研训练达到批判性阅读之前的阶段,但是创造性阅读的境界很难只通过训练达成。只有积极主动地去思考问题、把自己的背景知识与之联系起来,形成一个“故事”,并能够自己书写自己的“故事”时,才能在学术社区建立认同。



本次科研训练中,我希望同学们能够带着这十个问题去阅读文章,能够筛掉无用的信息、让真正有用的信息被构筑到自己的认知模型中,真正掌握这种科研思维模式。从机器学习的信息瓶颈(Information Bottleneck) 的类比看,这个过程就是让你的思维认知模式经过这十个问题模板形成的一个信息瓶颈而打造成型。



带着10个问题去阅读和思考


1. 这篇文章究竟讲了什么问题?比方说你设计一个算法,它的 input 和 output 是什么?



2. 这个问题的性质是什么?是一个新的问题吗?如果是一个新问题,它的重要性何在?如果不完全是一个新问题,那为什么它“仍然重要”?



我在西安交大念书的时候,沈老师曾经给我们做过一个演讲,其中有一句话令我印象深刻——“一流的研究员发现新问题”。发现有意义、有挑战性的新问题,实际上是一个研究最大的贡献。但毕竟学术领域内人才济济,很多人没有机会发现新问题,所以很多文章致力于回答第二个问题——为什么这个问题仍然值得去研究?


3. 这篇文章致力于证明什么假设?接受过深度科研训练的人都知道所有研究其实都是从科学假设开始的。从 12 年开始,计算机视觉领域的很多研究员认为这是一门实证科学(experimental science),即需要提出假说并通过实验去验证。



4. 有哪些与这篇文章相关的研究?这一领域有哪些关键人物?



大家做研究、读文章时,要了解这个方向的重要工作和从事相关研究的关键人员信息,才能把问题的来龙去脉搞清楚。让领域内的人们认识你、了解你的工作,你才能慢慢地被更多的专家、同行认可。我一直认为作为一名研究员要足够“八卦”,知道领域内哪些人做什么样的事情。


5. 这篇文章提出的问题解决方案中,核心贡献是什么?



6. 实验是如何设计的?计算机视觉研究中,实验设计的重要性不言而喻。但在写 paper 的过程中,实验的表现不是最重要的,关键是如何通过实验去支撑每个假说。


7. 实验是在什么样的数据集基础上运行的?科学研究结果应当是可以量化、可以复现的,读文章的人是否能接触到文中所用的数据集?



8. 实验结果能否有力地支持假设?如果一篇文章提出的假设并没有被实验或者理论完美支撑的话,多半不是一篇好的文章。


9. 这篇文章的贡献是什么?回答了前面 8 个问题之后,第9个问题的答案也呼之欲出了。你应当试着用自己的语言总结出来。



10. 下一步可以做什么?这是非常关键的一个问题,也决定了你今后能否在科研领域获得成功。在这篇文章的基础上,我们接下来能做什么?应该做什么?在科学研究的初期,导师会给你方向上的指导,但作为一名独立的研究员,你应该独立地回答这个问题。



以上就是这十个希望大家回答的问题。本次训练可以带领大家达到批判性阅读的阶段,但要最终达到创造性阅读的阶段,还需要你有“T 字型”的知识结构,即有足够广的知识储备,在某个方向钻得足够深。我们希望你也能提出自己的“10 个问题”,写出你自己的故事。这种多样性,正是科研领域健康发展的关键。



两位老师的精彩演讲是否令你心潮澎湃、充满干劲呢?“You are how you read”,阅读文章不仅是大家在科研道路上进步的必由之路,也能使我们的心智不断成长,认知模型和思维方式不断完善。这个暑假,让我们一起在科研之路上乘风破浪吧!




\section{陈怡然:如何在读研的道路上快速失败}

1

为自己找出一堆把其他个人兴趣爱好或者其他事情放在学习之上的理由:比如人活着就是为了享受生活,要事业家庭平衡之类(说真的,这是人生观问题。也许一开始你就根本不需要/不应该读研);

2

认为很多必须要做的事情或者要解决的问题拖到最后会不了了之、奇迹般迎刃而解、或者导师会忘(这怎么可能);

3

为导师不经常找自己而沾沾自喜(其实大部分时候那只是因为他觉得不值得在你身上瞎耽误功夫);

4

花很多时间计算到底怎么做才能快速达到最低毕业要求赶紧毕业(易经:取法乎上,仅得其中;取法乎中,仅得其下);

5

每次都指望从师兄师姐那里临时榨点什么以混过和导师的1-on-1(以为导师和师兄师姐傻啊?);

6

总觉得别人做的东西简单容易上手好发文章,不断在不同的题目中跳来跳去(世上那有什么容易的东西,只有你看不到的努力);

7

对导师和同学的建议与帮助出于本能的拒绝并试图证明别人是错的(自尊/自信/自负/自卑大多数时候其实我们自己不是那么容易分的清楚,但别人的善心只会offer一次);

8

总觉得自己比别人聪明,研究做不出来或者文章总不中只是一时粗心或者运气不好(不断重复发生的事情其实就是规律);

9

在每次觉得做不出来、或者做不完的时候玩消失,或者对更大、更复杂的任务说No(盖章:此人难当大任);

10

每次都刚刚做到最低要求,把剩余的工作扔给导师或者合作的同学,并为自己节省下的时间和努力沾沾自喜(你可能还没认识到别人对你的支持永远和你自己的付出成正比这一事实);

11

觉得自己很努力了,但是却怎么也达不到目标或者导师要求(如果确实不是你还有没有挤出来的时间,而你也不愿意承认自己确实不行,那么就多花点时间观察别人学习的方法和效率吧);

12

开始承认自己确实不行,并把全部精力放在劝说老师降低标准放自己毕业上(你可能没理解这件事情的难度在于你要求他降低的是对所有人的标准,而不单单是对你自己的);

13

觉得导师是个傻X(这个其实也不是没有可能。强扭的瓜不甜,我支持你用脚投票。或许到时双方都会大大的松了一口气,多年以后江湖再见还能把酒言欢)。

作者 | 陈怡然介绍 | 杜克大学电子与计算机工程系副教授,杜克进化智能研究中心主任,美国自然科学基金委新型可持续智能计算产学合作中心主任存储、类脑计算与深度学习专家,IEEE Fellow。



\subsection{ 如果讓我重做一次研究生}
%  http://homepage.ntu.edu.tw/~ylwang2008/a-wangvonsen.pdf

王汎森 院士
中央研究院歷史語言研究所(2005)

這個題目我非常喜歡,因為這個題目,對大家多少都有實際的幫助。如果
下次我必須再登台演講,我覺得這個題目還可以再發揮一兩次。我是台大歷史研
究所畢業的,所以我的碩士是在台大歷史研究所,我的博士是在美國普林斯頓大
學取得的。我想在座的各位有碩士、有博士,因此我以這兩個階段為主,把我的
經驗呈現給各位。



我從來不認為我是位有成就的學者,我也必須跟各位坦白,我為了要來做
這場演講,在所裡碰到剛從美國讀完博士回來的同事,因為他們剛離開博士生的
階段,比較有一些自己較獨特的想法,我就問他:「如果你講這個問題,準備要
貢獻什麼?」結合了他們的意見,共同醞釀了今天的演講內容,因此這裡面不全
是我一個人的觀點。雖然我的碩士論文和博士論文都出版了,但不表示我就是一
個成功的研究生,因為我也總還有其他方面仍是懵懵懂懂。我的碩士論文是二十
年前時報出版公司出版的,我的博士論文是英國劍橋大學出版的。你說有特別好
嗎?我不敢亂說。我今天只是綜合一些經驗,提供大家參考。

一、研究生與大學生的區別

首先跟大家說明一下研究生和大學生的區別。大學生基本上是來接受學
問、接受知識的,然而不管是對於碩士時期或是博士時期的研究而言,都應該準
備要開始製造新的知識,我們在美國得到博士學位時都會領到看不懂的畢業證
書,在一個偶然的機會下,我問了一位懂拉丁文的人,上面的內容為何?他告訴
我:「裡頭寫的是恭喜你對人類的知識有所創新,因此授予你這個學位。」在中
國原本並沒有博碩士的學歷,但是在西方他們原來的用意是,恭賀你已經對人類
普遍的知識有所創新,這個創新或大或小,都是對於普遍的知識有所貢獻。這個
創新不會因為你做本土與否而有所不同,所以第一個我們必須要很用心、很深刻
的思考,大學生和研究生是不同的。

(一)選擇自己的問題取向,學會創新
你一旦是研究生,你就已經進入另一個階段,不只是要完全樂在其中,更
要從而接受各種有趣的知識,進入製造知識的階段,也就是說你的論文應該有所
創新。由接受知識到創造知識,是身為一個研究生最大的特色,不僅如此,還要
體認自己不再是個容器,等著老師把某些東西倒在茶杯裡,而是要開始逐步發展
和開發自己。做為研究生不再是對於各種新奇的課照單全收,而是要重視問題取
向的安排,就是在碩士或博士的階段裡面,所有的精力、所有修課以及讀的書裡
面都應該要有一個關注的焦點,而不能像大學那般漫無目標。大學生時代是因為
你要盡量開創自己接受任何東西,但是到了碩士生和博士生,有一個最終的目
的,就是要完成論文,那篇論文是你個人所有武功的總集合,所以這時候必須要
有個問題取向的學習。


(二)嘗試跨領域研究,主動學習

提出一個重要的問題,跨越一個重要的領域,將決定你未來的成敗。我也
在台大和清華教了十幾年的課,我常常跟學生講,選對一個領域和選對一個問題
是成敗的關鍵,而你自己本身必須是帶著問題來探究無限的學問世界,因為你不
再像大學時代一樣氾濫無所歸。所以這段時間內,必須選定一個有興趣與關注的
主題為出發點,來探究這些知識,產生有機的循環。由於你是自發性的對這個問
題產生好奇和興趣,所以你的態度和大學部的學生是截然不同的,你慢慢從被動
的接受者變成是一個主動的探索者,並學會悠游在這學術的領域。

我舉一個例子,我們的中央研究院院長李遠哲先生,得了諾貝爾獎。他曾
經在中研院的週報寫過幾篇文章,在他的言論集裡面,或許各位也可以看到,他
反覆提到他的故事。他是因為讀了一個叫做馬亨教授的教科書而去美國柏克萊大
學唸書,去了以後才發現,這個老師只給他一張支票,跟他說你要花錢你盡量用,
但是從來不教他任何東西。可是隔壁那個教授,老師教很多,而且每天學生都是
跟著老師學習。他有一次就跟那個老師抱怨:「那你為什麼不教我點東西呢?」
那個老師就說:「如果我知道結果,那我要你來這邊唸書做什麼?我就是因為不
知道,所以要我們共同探索一個問題、一個未知的領域。」他說其實這兩種教法
都有用處,但是他自己從這個什麼都不教他,永遠碰到他只問他「有沒有什麼新
發現」的老師身上,得到很大的成長。所以這兩方面都各自蘊含深層的道理,沒
有所謂的好壞,但是最好的方式就是將這兩個方式結合起來。我為什麼講這個故
事呢?就是強調在這個階段,學習是一種「self-help」,並且是在老師的引導下學
習「self-help」,而不能再像大學時代般,都是純粹用聽的,這個階段的學習要基
於對研究問題的好奇和興趣,要帶著一顆熱忱的心來探索這個領域。

然而研究生另外一個重要的階段就是 Learn how to learn,不只是學習而已,
而是學習如何學習,不再是要去買一件很漂亮的衣服,而是要學習拿起那一根
針,學會繡出一件漂亮的衣服,慢慢學習把目標放在一個標準上,而這一個標準
就是你將來要完成碩士或博士論文。如果你到西方一流的大學去讀書,你會覺得
我這一篇論文可能要和全世界做同一件問題的人相比較。我想即使在台灣也應該
要有這樣的心情,你的標準不能單單只是放在旁邊幾個人而已,而應該是要放在
領域的普遍人裡面。你這篇文章要有新的東西,才算達到的標準,也才符合到我
們剛剛講到那張拉丁文的博士證書上面所講的,有所貢獻與創新。


二、一個老師怎麼訓練研究生

第二個,身為老師你要怎麼訓練研究生。我認為人文科學和社會科學的訓
練,哪怕是自然科學的訓練,到研究生階段應該更像師徒制,所以來自個人和老
師、個人和同儕間密切的互動和學習是非常重要的,跟大學部坐在那邊單純聽
課,聽完就走人是不一樣的,相較之下你的生活應該要和你所追求的知識與解答
相結合,並且你往後的生活應該或多或少都和這個探索有相關。

(一)善用與老師的夥伴關係,不斷 Research
我常說英文 research 這個字非常有意義,search 是尋找,而 research 是再尋找,
所以每個人都要 research,不斷的一遍一遍再尋找,並進而使你的生活和學習成
為一體。中國近代兵學大師蔣百里在他的兵學書中曾說:「生活條件要跟戰鬥條
件一致,近代歐洲凡生活與戰鬥條件一致者強,凡生活與戰鬥條件不一致者弱。」
我就是藉由這個來說明研究生的生活,你的生活條件與你的戰鬥條件要一致,你
的生活是跟著老師與同學共同成長的,當中你所聽到的每一句話,都可能帶給你
無限的啟發。
回想當時我在美國唸書的研究生生活,只要隨便在樓梯口碰到任何一個人,
他都有辦法幫忙解答你語言上的困難,不管是英文、拉丁文、德文、希臘文……
等。所以能幫助解決問題的不單只是你的老師,還包括所有同學以及學習團體。
你的學習是跟生活合在一起的。當我看到有學生呈現被動或是懈怠的時候,我就
會用毛澤東的「革命不是請客吃飯!」來跟他講:「作研究生不是請客吃飯。」 

(二)藉由大量閱讀和老師提點,進入研究領域

怎樣進入一個領域最好,我個人覺得只有兩條路,其中一條就是讓他不停的
唸書、不停的報告,這是進入一個陌生的領域最快,又最方便的方法,到最後不
知不覺學生就會知道這個領域有些什麼,我們在不停唸書的時候常常可能會沉溺
在細節裡不能自拔,進而失去全景,導致見樹不見林,或是被那幾句英文困住,
而忘記全局在講什麼。藉由學生的報告,老師可以講述或是釐清其中的精華內
容,經由老師幾句提點,就會慢慢打通任督二脈,逐漸發展一種自發學習的能力,
同時也知道碰到問題可以看哪些東西。就像是我在美國唸書的時候,我修過一些
我完全沒有背景知識的國家的歷史,所以我就不停的唸書、不停的逼著自己吸
收,而老師也只是不停的開書目,運用這樣的方式慢慢訓練,有一天我不再研究
它時,我發現自己仍然有自我生產及蓄發的能力,因為我知道這個學問大概是什
麼樣的輪廓,碰到問題也有能力可以去查詢相關的資料。所以努力讓自己的學習
產生自發的延展性是很重要的。 

(三)循序漸進地練習論文寫作

到了碩士或博士最重要的一件事,是完成一篇學位論文,而不管是碩士或博
士論文,其規模都遠比你從小學以來所受的教育、所要寫的東西都還要長得多,
雖然我不知道教育方面的論文情況是如何,但是史學的論文都要寫二、三十萬
字,不然就是十幾二十萬字。寫這麼大的一個篇幅,如何才能有條不紊、條理清
楚,並把整體架構組織得通暢可讀?首先,必須要從一千字、五千字、一萬字循
序漸進的訓練,先從少的慢慢寫成多的,而且要在很短的時間內訓練到可以從一
萬字寫到十萬字。這麼大規模的論文誰都寫得出來,問題是寫得好不好,因為這
麼大規模的寫作,有這麼許多的註腳,還要注意首尾相映,使論述一體成型,而
不是散落一地的銅錢;是一間大禮堂,而不是一間小小分割的閣樓。為了完成一
個大的、完整的、有機的架構模型,必須要從小規模的篇幅慢慢練習,這是一個
最有效的辦法。

因為受電腦的影響,我發現很多學生寫文章能力都大幅下降。寫論文時很重
要的一點是,文筆一定要清楚,不要花俏、不必漂亮,「清楚」是最高指導原則,
經過慢慢練習會使你的文筆跟思考產生一致的連貫性。我常跟學生講不必寫的花
俏,不必展現你散文的才能,因為這是學術論文,所以關鍵在於要寫得非常清楚,
如果有好的文筆當然更棒,但那是可遇不可求的,文彩像個人的生命一樣,英文
叫 style,style 本身就像個人一樣帶有一點點天生。因此最重要的還是把內容陳述
清楚,從一萬字到最後十萬字的東西,都要架構井然、論述清楚、文筆清晰。

我在唸書的時候,有一位歐洲史、英國史的大師 Lawrence Stone,他目前已
經過世了,曾經有一本書訪問十位最了不起的史學家,我記得他在訪問中說了一
句非常吸引人注意的話,他說他英文文筆相當好,所以他一輩子沒有被退過稿。
因此文筆清楚或是文筆好,對於將來文章可被接受的程度有舉足輕重的地位。內
容非常重要,有好的表達工具更是具有加分的作用,但是這裡不是講究漂亮的
style,而是論述清楚。 


三、研究生如何訓練自己

(一)嘗試接受挑戰,勇於克服

研究生如何訓練自己?就是每天、每週或每個月給自己一個挑戰,要每隔一
段時間就給自己一個挑戰,挑戰一個你做不到的東西,你不一定要求自己每次都
能順利克服那個挑戰,但是要努力去嘗試。我在我求學的生涯中,碰到太多聰明
但卻一無所成的人,因為他們很容易困在自己的障礙裡面,舉例來說,我在普林
斯頓大學碰到一個很聰明的人,他就是沒辦法克服他給自己的挑戰,他就總是東
看西看,雖然我也有這個毛病,可是我會定期給我自己一個挑戰,例如:我會告
訴自己,在某一個期限內,無論如何一定要把這三行字改掉,或是這個禮拜一定
要把這篇草稿寫完,雖然我仍然常常寫不完,但是有這個挑戰跟沒這個挑戰是不
一樣的,因為我挑戰三次總會完成一次,完成一次就夠了,就足以表示克服了自
己,如果覺得每一個禮拜的挑戰,可行性太低,可以把時間延長為一個月的挑戰,
去挑戰原來的你,不一定能做到的事情。不過也要切記,碩士生是剛開始進入這
一個領域的新手,如果一開始問題太小,或是問題大到不能控制,都會造成以後
研究的困難。 

(二)論文的寫作是個訓練過程,不能苛求完成精典之作

各位要記得我以前的老師所說的一句話:「碩士跟博士是一個訓練的過程,
碩士跟博士不是寫經典之作的過程。」我看過很多人,包括我的親戚朋友們,他
之所以沒有辦法好好的完成碩士論文,或是博士論文,就是因為他把它當成在寫
經典之作的過程,雖然事實上,很多人一生最好的作品就是碩士論文或博士論
文,因為之後的時間很難再有三年或六年的時間,沉浸在一個主題裡反覆的耕
耘,當你做教授的時候,像我今天被行政纏身,你不再有充裕的時間好好探究一
個問題,尤其做教授還要指導學生、上課,因此非常的忙碌,所以他一生最集中
又精華的時間,當然就是他寫博士、或是碩士論文的時候,而那一本成為他一生
中最重要的著作也就一點都不奇怪了。 

但不一定要刻意強求,要有這是一個訓練過程的信念,應該清楚知道從哪裡
開始,也要知道從哪裡放手,不要無限的追下去。當然我不是否認這個過程的重
要性,只是要調整自己的心態,把論文的完成當成一個目標,不要成為是一種的
心理障礙或是心理負擔。這方面有太多的例子了,我在普林斯頓大學唸書的時
候,那邊舊書攤有一位非常博學多文的舊書店老闆,我常常讚嘆的對他說:「你
為什麼不要在大學做教授。」他說:「因為那篇博士論文沒有寫完。」原因在於
他把那個博士論文當成要寫一本經典,那當然永遠寫不完。如果真能寫成經典那
是最好,就像美麗新境界那部電影的男主角 John Nash 一樣,一生最大的貢獻就
是博士那二十幾頁的論文,不過切記不要把那個當作是目標,因為那是自然而然
形成的,應該要堅定的告訴自己,所要完成的是一份結構嚴謹、論述清楚與言之
有物的論文,不要一開始就期待它是經典之作。如果你期待它是經典之作,你可
能會變成我所看到的那位舊書攤的老闆,至於我為什麼知道他有那麼多學問,是
因為那時候我在找一本書,但它並沒有在舊書店裡面,不過他告訴我:「還有很
多本都跟他不相上下。」後來我對那個領域稍稍懂了之後,證明確實如他所建議
的那般。一個舊書店的老闆精熟每一本書,可是他就是永遠無法完成,他夢幻般
的學位論文,因為他不知道要在哪裡放手,這一切都只成為空談。

(三)論文的正式寫作
1.學習有所取捨 

到了寫論文的時候,要能取也要能捨,因為現在資訊爆炸,可以看的書太多,
所以一定要建構一個屬於自己的知識樹,首先,要有一棵自己的知識樹,才能在
那棵樹掛相關的東西,但千萬不要不斷的掛不相關的東西,而且要慢慢的捨掉一
些掛不上去的東西,再隨著你的問題跟關心的領域,讓這棵知識樹有主幹和枝
葉。然而這棵知識樹要如何形成?第一步你必須對所關心的領域中,有用的書籍
或是資料非常熟悉。 

2.形成你的知識樹
我昨天還請教林毓生院士,他今年已經七十幾歲了,我告訴他我今天要來作
演講,就問他:「你如果講這個題目你要怎麼講?」他說:「只有一點,就是那
重要的五、六本書要讀好幾遍。」因為林毓生先生是海耶克,還有幾位近代思想
大師在芝加哥大學的學生,他們受的訓練中很重要的一部份是精讀原典。這句話
很有道理,雖然你不可能只讀那幾本重要的書,但是那五、六本書將逐漸形成你
知識樹的主幹,此後的東西要掛在上面,都可以參照這一個架構,然後把不相干
的東西暫放一邊。生也有涯,知也無涯,你不可能讀遍天下所有的好書,所以要
學習取捨,了解自己無法看遍所有有興趣的書,而且一但看遍所有有興趣的書,
很可能就會落得普林斯頓街上的那位舊書店的老闆一般,因為閱讀太多不是自己
所關心的領域的知識,它對於你來說只是一地的散錢。 


3.掌握工具

在這個階段一定要掌握語文與合適的工具。要有一個外語可以非常流暢的閱
讀,要有另外一個語文至少可以看得懂文章的標題,能學更多當然更好,但是至
少要有一個語文,不管是英文、日文、法文……等,一定要有一個語文能夠非常
流暢的閱讀相關書籍,這是起碼的前提。一旦這個工具沒有了,你的視野就會因
此大受限制,因為語文就如同是一扇天窗,沒有這個天窗你這房間就封閉住了。
為什麼你要看得懂標題?因為這樣才不會有重要的文章而你不知道,如果你連標
題都看不懂,你就不知道如何找人來幫你或是自己查相關的資料。其他的工具,
不管是統計或是其他的任何工具,你也一定要多掌握,因為你將來沒有時間再把
這樣的工具學會。

4.突破學科間的界線

應該要把跨學科的學習當作是一件很重要的事,但是跨學科涉及到的東西必
須要對你這棵知識樹有助益,要學會到別的領域稍微偷打幾槍,到別的領域去攝
取一些概念,對於本身關心的問題產生另一種不同的啟發,可是不要氾濫無所
歸。為什麼要去偷打那幾槍?近幾十年來,人們發現不管是科學或人文,最有創
新的部份是發生在學科交會的地方。為什麼會如此?因為我們現在的所有學科大
部分都在西方十九世紀形成的,而中國再把它轉借過來。十九世紀形成這些知識
學科的劃分的時候,很多都帶有那個時代的思想跟學術背景,比如說,中研院的
李院長的專長就是物理化學,他之所以得諾貝爾獎就是他在物理和化學的交界處
做工作。像諾貝爾經濟獎,這二十年來所頒的獎,如果在傳統的經濟學獎來看就
是旁門走道,古典經濟學豈會有這些東西,甚至心理學家也得諾貝爾經濟獎,連
John Nash 這位數學家也得諾貝爾經濟獎,為什麼?因為他們都在學科的交界上,
學科跟學科、平台跟平台的交界之處有所突破。在平台本身、在學科原本最核心
的地方已經 search 太多次了,因此不一定能有很大的創新,所以為什麼跨領域學
習是一件很重要的事情。

常常一篇碩士論文或博士論文最重要、最關鍵的,是那一個統攝性的重要概
念,而通常你在本學科裡面抓不到,是因為你已經泡在這個學科裡面太久了,你
已經拿著手電筒在這個小倉庫裡面照來照去照太久了,而忘了還有別的東西可以
更好解釋你這些材料的現象,不過這些東西可遇而不可求。John Nash 這一位數
學家為什麼會得諾貝爾數學獎?為什麼他在賽局理論的博士論文,會在數十年之
後得諾貝爾經濟獎?因為他在大學時代上經濟學導論的課,所以他認為數學可以
用在經濟方面來思考,而這個東西在一開始,他也沒有想到會有這麼大的用處。
他是在數學和經濟學的知識交界之處做突破。有時候在經濟學這一個部分沒有大
關係,在數學的這一個部分也沒有大關係,不過兩個加在一起,火花就會蹦出來。 
5.論文題目要有延展性
對一個碩士生或博士生來說,如果選錯了題目,就是失敗,題目選對了,還
有百分之七十勝利的機會。這個問題值得研一、博一的學生好好思考。你的第一
年其實就是要花在這上面,你要不斷的跟老師商量尋找一個有意義、有延展性的
問題,而且不要太難。我在國科會當過人文處長,當我離開的時候,每次就有七
千件申請案,就有一萬四千個袋子,就要送給一萬四千個教授審查。我當然不可
能看那麼多,可是我有個重要的任務,就是要看申訴。有些申訴者認為:「我的
研究計畫很好,我的著作很好,所以我來申訴。」申訴通過的大概只有百分之十,
那麼我的責任就是在百分之九十未通過的案子正式判決前,再拿來看一看。有幾
個印象最深常常被拿出來討論的,就是這個題目不必再做了、這個題目本身沒有
發展性,所以使我更加確認選對一個有意義、有延展性、可控制、可以經營的題
目是非常重要的。
我的學生常常選非常難的題目,我說你千萬不要這樣,因為沒有人會仔細去
看你研究的困難度,對於難的題目你要花更多的時間閱讀史料,才能得到一點點
東西;要擠很多東西,才能篩選出一點點內容,所以你最好選擇一個難易適中的
題目。

我寫過好幾本書,我認為我對每一本書的花的心力都是一樣,雖然我寫任何
東西我都不滿意,但是在過程中我都絞盡腦汁希望把他寫好。目前為止很多人認
為我最好的書,是我二十幾歲剛到史語所那一年所寫的那本書。我在那本書花的
時間並不長,那本書的大部分的稿子,是我和許添明老師同時在當兵的軍營裡面
寫的,而且還是用我以前舊的筆記寫的。大陸這些年有許多出版社,反覆要求出
版我以前的書,尤其是這一本,我說:「不行。」因為我用的是我以前的讀書筆
記,我怕引文有錯字,因為在軍隊營區裡面隨時都要出操、隨時就要集合,手邊
又沒有書,怎麼可能好好的去核對呢?而如果要我重新校正一遍,又因為引用太
多書,實在沒有力氣校正。

為什麼舉這個例子呢?我後來想一想,那本書之所以比較好,可能是因為那
個題目可延展性大,那個題目波瀾起伏的可能性大。很多人都認為,我最好的書
應該是劍橋大學出的那一本,不過我認為我最好的書一定是用中文寫的,因為這
個語文我能掌握,英文我沒辦法掌握得出神入化。讀、寫任何語文一定要練習到
你能帶著三分隨意,那時候你才可以說對於這一個語文完全理解與精熟,如果你
還無法達到三分的隨意,就表示你還在摸索。

回到我剛剛講的,其實每一本書、每一篇論文我都很想把它寫好。但是有些
東西沒辦法寫好,為什麼?因為一開始選擇的題目不夠好。因此唯有選定題目以
後,你的所有訓練跟努力才有價值。我在這裡建議大家,選題的工作要儘早做,
所選的題目所要處理的材料最好要集中,不要太分散,因為碩士生可能只有三
年、博士生可能只有五年,如果你的材料太不集中,讀書或看資料可能就要花掉
你大部分的時間,讓你沒有餘力思考。而且這個題目要適合你的性向,如果你不
會統計學或討厭數字,但卻選了一個全都要靠統計的論文,那是不可能做得好。 
6.養成遵照學術格式的寫作習慣

另一個最基本的訓練,就是平時不管你寫一萬字、三萬字、五萬字都要養成
遵照學術規範的習慣,要讓他自然天成,就是說你論文的註腳、格式,在一開始
進入研究生的階段就要培養成為你生命中的一個部份,如果這個習慣沒有養成,
人家就會覺得這個論文不嚴謹,之後修改也要花很多時間,因為你的論文規模很
大,可能幾百頁,如果一開始弄錯了,後來再重頭改到尾,一定很耗時費力,因
此要在一開始就養成習慣,因為我們是在寫論文而不是在寫散文,哪一個逗點應
該在哪裡、哪一個書名號該在哪裡、哪一個地方要用引號、哪一個要什麼標點符
號,都有一定的規定,用中文寫還好,用英文有一大堆簡稱。在 1960 年代台灣
知識還很封閉的時候,有一個人從美國回來就說:「美國有個不得了的情形,因
為有一個人非常不得了。」有人問他為什麼不得了,他說:「因為這個人的作品
到處被引用。」他的名字就叫 ibid。所謂 ibid 就是同前作者,這個字是從拉丁文
發展出來的,拉丁文有一大堆簡稱,像 et. al.就是兩人共同編的。英文有一本 The
Chicago Manual of Style 就是專門說明這一些寫作規範。各位要儘早學會中英文的
寫作規範,慢慢練習,最後隨性下筆,就能寫出符合規範的文章。

7.善用圖書館

 圖書館應該是研究生階段最重要的地方,不必讀每一本書,可是要知道有哪些
書。我記得我做學生時,新進的書都會放在圖書館的牆上,而身為學生最重要的
事情,就是要把書名看一看。在某些程度上知道書皮就夠了,但是這仍和打電腦

是不一樣的,你要實際上熟悉一下那本書,摸一下,看一眼目錄。我知道現在從
電腦就可以查到書名,可是我還是非常珍惜這種定期去 browse 新到的書的感覺,
或去看看相關領域的書長成什麼樣子。中研院有一位院士是哈佛大學資訊教授,
他告訴我他在創造力最高峰的時候,每個禮拜都到他們資訊系圖書室裡,翻閱重
要的資訊期刊。所以圖書館應該是身為研究生的人們,最熟悉的地方。不過切記
不重要的不要花時間去看,你們生活在資訊氾濫的時代,跟我生長在資訊貧乏的
時代是不同的,所以生長在這一個時代的你,要能有所取捨。我常常看我的學生
引用一些三流的論文,卻引得津津有味,我都替他感到難過,因為我強調要讀有
用、有價值的東西。

8.留下時間,精緻思考

還要記得給自己保留一些思考的時間。一篇論文能不能出神入化、能不能引
人入勝,很重要的是在現象之上作概念性的思考,但我不是說一定要走理論的路
線,而是提醒大家要在一般的層次再提升兩三步,conceptualize 你所看到的東西。
真切去了解,你所看到的東西是什麼?整體意義是什麼?整體的輪廓是什麼?千
萬不要被枝節淹沒,雖然枝節是你最重要的開始,但是你一天總也要留一些時間
好好思考、慢慢沉澱。conceptualize 是一種非常難教的東西,我記得我唸書時,
有位老師信誓旦旦說要開一門課,教學生如何 conceptualize,可是從來都沒開成,
因為這非常難教。我要提醒的是,在被很多材料和枝節淹沒的時候,要適時跳出
來想一想,所看到的東西有哪些意義?這個意義有沒有廣泛連結到更大層面的知
識價值。

傅斯年先生來到台灣以後,同時擔任中央研究院歷史語言研究所的所長及台
大的校長。台大有個傅鐘每小時鐘聲有二十一響、敲二十一次。以前有一個人,
寫了一本書叫《鐘聲二十一響》,當時很轟動。他當時對這二十一響解釋是說:
因為台大的學生都很好,所以二十一響是歡迎國家元首二十一響的禮炮。不久前
我發現台大在每一個重要的古蹟下面豎一個銅牌,我仔細看看傅鐘下的解釋,才
知道原來是因為傅斯年當台大校長的時候,曾經說過一句話:「人一天只有二十
一個小時,另外三小時是要思考的。」所以才叫二十一響。我覺得這句話大有道
理,可是我覺得三小時可能太多,因為研究生是非常忙的,但至少每天要留個三
十分鐘、一小時思考,想一想你看到了什麼?學習跳到比你所看到的東西更高一
點的層次去思考。 

9.找到學習的楷模

我剛到美國唸書的時候,每次寫報告頭皮就重的不得了,因為我們的英文報
告三、四十頁,一個學期有四門課的話就有一百六十頁,可是你連註腳都要從頭
學習。後來我找到一個好辦法,就是我每次要寫的時候,把一篇我最喜歡的論文
放在旁邊,雖然他寫的題目跟我寫的都沒關係,不過我每次都看他如何寫,看看
他的注腳、讀幾行,然後我就開始寫。就像最有名的男高音 Pavarotti 唱歌劇的時
候都會捏著一條手帕,因為他說:「上舞台就像下地獄,太緊張了。」他為了克
服緊張,他有習慣性的動作,就是捏著白手帕。我想當年那一篇論文抽印本就像
是我的白手帕一樣,能讓我開始好好寫這篇報告,我學習它裡面如何思考、如何
構思、如何照顧全體、如何用英文作註腳。好好的把一位大師的作品讀完,開始
模仿和學習他,是入門最好的方法,逐步的,你也開始寫出自己的東西。我也常
常鼓勵我的學生,出國半年或是一年到國外看看。像現在國科會有各式各樣的機
會,可以增長眼界,可以知道現在的餐館正在賣些什麼菜,回來後自己要作菜也
才知道要如何著手。

四、用兩條腿走路,練習培養自己的興趣

 最後還有一點很重要的,就是我們的人生是兩隻腳,我們不是靠一隻腳走
路。做研究生的時代,固然應該把所有的心思都放在學業上,探索你所要探索的
那些問題,可是那只是你的一隻腳,另外還有一隻腳是要學習培養一、兩種興趣。
很多人後來會發現他的右腳特別肥重(包括我自己在內),也就是因為忘了培養
左腳。很多很有名的大學者最後都陷入極度的精神困擾之中,就是因為他只是培
養他的右腳,他忘了培養他的左腳,他忘了人生用兩隻腳走路,他少了一個小小
的興趣或嗜好,用來好好的調解或是排遣自己。

 去年夏天,香港《亞洲週刊》要訪問我,我說:「我不想接受訪問,我不
是重要的人。」可是後來他們還是把一個簡單的對話刊出來了,裡面我只記得講
了一段話:做一個研究生或一個學者,有兩個感覺最重要--責任感與罪惡感。你
一定要有很大的責任感,去寫出好的東西,如果責任感還不夠強,還要有一個罪
惡感,你會覺得如果今天沒有好好做幾個小時的工作的話,會有很大的罪惡感。
除非是了不得的天才,不然即使愛因斯坦也是需要很努力的。很多很了不得的
人,他只是把所有的努力集中在一百頁裡面,他花了一千小時和另外一個人只花
了十個小時,相對於來說,當然是那花一千個小時所寫出來的文章較好。所以為
什麼說要趕快選定題目?因為如果太晚選定一個題目,只有一年的時間可以好好
耕耘那個題目,早點選定可以有二、三年耕耘那個題目,是三年做出的東西好,
還是一年的東西好?如果我們的才智都一樣的話,將三年的努力與思考都灌在上
面,當然比一年還要好。 

五、營造卓越的大學,分享學術的氛圍

現在很多人都在討論,何謂卓越的大學?我認為一個好的大學,學校生活的
一大部份,以及校園的許多活動,直接或間接都與學問有關,同學在咖啡廳裡面
談論的,直接或間接也都會是學術相關的議題。教授們在餐廳裡面吃飯,談的是
「有沒有新的發現」?或是哪個人那天演講到底講了什麼重要的想法?一定是沉
浸在這種氛圍中的大學,才有可能成為卓越大學。那種交換思想學識、那種互相
教育的氣氛不是花錢就有辦法獲得的。我知道錢固然重要,但不是唯一的東西。
一個卓越的大學、一個好的大學、一個好的學習環境,表示裡面有一個共同關心
的焦點,如果沒有的話,這個學校就不可能成為好的大學。

\section{给研究生的十个建议}

美国芝加哥伊利诺伊大学生理学家德拉内罗(Primal de Lanerolle), 发表获得成功的
研究所生涯最重要的十大窍门。

中文版: \url{http://homepage.ntu.edu.tw/~ylwang2008/image/recruiters&academia.jpg}
英文版: \url{http://homepage.ntu.edu.tw/~ylwang2008/image/recruiters&academia.gif}
   \url{https://www.nature.com/articles/nj7024-442b}

确保你的选择能够提升你在研究工作与职场的前景

十  开始

不要用大学部逃课的方式,来选研究所的课程。
读大学是为了获得更广泛的教育,读研究生院是为了获得学位。

九 选指导教授
 指导教授就像你的配偶,选一位不论在学术与人品都让你尊敬的人。
 你的系所就像你配偶的家庭,挑一个严格且具挑战性的地方
 
八 认识你自己
认清你的优缺点,和你的好恶。成功的人会专注于发挥所长,减少缺点。

七 写一份引以为傲的论文
你一辈子可能只会写一次博士论文,让你所付出的心血值得。

六 做好的科学

脚踏实地、控制良好的科学非常重要。
不顺利的时候,她是能够帮你摆脱麻烦的唯一办法;
顺利的时候,它可以组织你做傻事。
挑剔你的数据,如同你挑剔其他人的数据一样。

五  最好的科学家是艺术家
提出的问题好不好,会跟你用以提问的技术专业,收到同等的批判。
精巧优秀的假设需要想象力、洞察力以及见人所未见之处的能力

四  没有胆量,就没有荣耀
你必须要有勇气才能有创意。培养不怕失败的自信,
问大问题并设定高的标准。
此外,知道自己有时是运气好。 如果你不认知这点,幸运就会失去意义。

三 数据是权威
只有数据是神圣的,假设与信仰都不是。
一丝不苟的分析你的数据,不要只看到你希望看到的结果。
一些最有意思的发现,是哪些我们不预期会看到的东西。
不能因为你不想看到,或者它使你困惑,而拒绝某些数据。

二 玩的愉快

成功需要大量的努利和投入。 你必须喜爱并且 享受这个过程。
试着让兴趣成为你的职业。

一 记住你在追求真实

你的目标是加入学者的社群,学这存在的唯一目的是求真。
真实难以捉摸,但科学方法提供了最客观的标准来评判真实。






