\chapter{科研事件}

\section{ISCA}

中国在体系结构旗舰会议ISCA的第一次突破是在1981年。当年有两篇来自中国的论文被录用,一篇是计算所夏培肃团队,另一篇是复旦大学朱传琪和同事。
1993年,高庆狮院士发表了中国的第三篇ISCA,介绍孙子定理在内存地址映射中的应用。
第四篇ISCA则要到2007年了,国防科大团队的飞腾FT64流处理器。
此后10年是井喷期,国内发表约20篇,但仍然集中在不超过5所大学和研究机构。


